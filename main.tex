\documentclass{article}
\usepackage[left=1in,right=1in]{geometry}
\usepackage{subfiles}
\usepackage{amsmath, amssymb, stmaryrd, verbatim} % math symbols
\usepackage{amsthm} % thm environment
\usepackage{mdframed} % Customizable Boxes
\usepackage{hyperref,nameref,cleveref,enumitem} % for references, hyperlinks
\usepackage[dvipsnames]{xcolor} % Fancy Colours
\usepackage{mathrsfs} % Fancy font
\usepackage{bbm} % mathbb numerals
\usepackage[bbgreekl]{mathbbol} % for bb Delta
\usepackage{tikz, tikz-cd, float} % Commutative Diagrams
\usetikzlibrary{decorations.pathmorphing} % for squiggly arrows in tikzcd
\usepackage{perpage}
\usepackage{parskip} % So that paragraphs look nice
\usepackage{ifthen,xargs} % For defining better commands
\usepackage[T1]{fontenc}
\usepackage[utf8]{inputenc}
\usepackage{tgpagella}
\usepackage{cancel}

% Bibliography
% \usepackage{url}
\usepackage[backend=biber,
            isbn=false,
            doi=false,
            giveninits=true,
            style=alphabetic,
            useprefix=true,
            maxcitenames=4,
            maxbibnames=4,
            sorting=nyt,
            citestyle=alphabetic]{biblatex}

% Shortcuts

% % Local to this project
\setcounter{secnumdepth}{4}
\renewcommand\labelitemi{--} % Makes itemize use dashes instead of bullets
\newcommand{\bbrkt}[1]{\llbracket #1 \rrbracket}
\newcommand{\IND}{\mathrm{Ind}}
\newcommand{\AFF}{\mathrm{Aff}}
\newcommand{\FSCH}{\mathrm{fSch}}
\newcommand{\SCH}{\mathrm{Sch}}
\DeclareMathOperator{\SPEC}{Spec}
\DeclareMathOperator{\QCOH}{QCoh}
\DeclareMathOperator{\INDCOH}{IndCoh}
\newcommand{\DR}{\mathrm{dR}}
\newcommand{\SYM}{\mathrm{Sym}}
\newcommand{\HOM}{\mathrm{Hom}}
\newcommand{\MAP}{\mathrm{Map}}
\newcommand{\CALG}{\mathrm{cAlg}}
\DeclareMathOperator{\GL}{GL}
\DeclareMathOperator{\LIE}{Lie}
\newcommand{\AD}{\mathrm{Ad}}
\DeclareMathOperator{\MAX}{Max}
\DeclareMathOperator{\FROB}{Frob}
\newcommand{\SEP}{\mathrm{sep}}
\newcommand{\LOC}{\mathrm{Loc}}
\DeclareMathOperator{\PERV}{Perv}
\newcommand{\RED}{\mathrm{red}}
\newcommand{\ZAR}{\mathrm{Zar}}
\newcommand{\FPPF}{\mathrm{fppf}}
\newcommand{\ET}{\text{ét}}
\DeclareMathOperator{\SPF}{Spf}
\newcommand{\CTS}{\mathrm{cts}}
\newcommand{\INF}{\mathrm{inf}}
\DeclareMathOperator{\STK}{Stk}
\DeclareMathOperator{\PSTK}{PStk}
\DeclareMathOperator{\SSET}{sSet}
\DeclareMathOperator{\BUN}{Bun}
\DeclareMathOperator{\PSH}{PSh}
\DeclareMathOperator{\SH}{Sh}
\newcommand{\FPQC}{\mathrm{fpqc}}
\DeclareMathOperator{\SPH}{Sph}
\newcommand{\FT}{\mathrm{ft}}
\DeclareMathOperator{\CRYS}{Crys}
\DeclareMathOperator{\GR}{Gr}
\DeclareMathOperator{\REP}{Rep}
\newcommand{\HITCH}{\mathrm{Hitch}}
\newcommand{\CRIT}{\mathrm{crit}}
\DeclareMathOperator{\LOCSYS}{LocSys}
\newcommand{\GRPD}{\mathrm{Grpd}}
\DeclareMathOperator{\FUN}{Fun}
\newcommand{\HECKE}{\mathrm{Hecke}}
\newcommand{\TRIV}{\mathrm{Triv}}
\newcommand{\REGLUE}{\mathrm{ReGlue}}
\newcommand{\LVL}{\mathrm{lvl}}
\newcommand{\PT}{\mathrm{pt}}
\DeclareMathOperator{\COH}{Coh}
\DeclareMathSymbol\bbDe  \mathord{bbold}{"01} % blackboard Delta
\newcommand{\CL}{\mathrm{cl}}
\newcommand{\OPER}{\operatorname{Oper}}
\DeclareMathOperator{\PERF}{Perf}
\DeclareMathOperator{\INV}{Inv}
\newcommand{\PIC}{\mathrm{Pic}}
\newcommand{\RAT}{\mathrm{Rat}}
\newcommand{\DIV}{\mathrm{Div}}
\newcommand{\SPA}{\mathrm{Spa}\,}
\newcommand{\SPD}{\mathrm{Spd}\,}
\newcommand{\LA}{\mathrm{la}}
\newcommand{\SNORM}{\mathrm{SNorm}}
\newcommand{\NORM}{\mathrm{Norm}}
\newcommand{\BAN}{\mathrm{Ban}}
\newcommand{\CYC}{\mathrm{cyc}}
\newcommand{\FRAC}{\mathrm{Frac}}


% % Misc
\newcommand{\brkt}[1]{\left(#1\right)}
\newcommand{\sqbrkt}[1]{\left[#1\right]}
\newcommand{\dash}{\text{-}}

% % Logic
\renewcommand{\implies}{\Rightarrow}
\renewcommand{\iff}{\Leftrightarrow}
\newcommand{\limplies}{\Leftarrow}
\newcommand{\NOT}{\neg\,}
\newcommand{\AND}{\, \land \,}
\newcommand{\OR}{\, \lor \,}
\newenvironment{forward}{($\implies$)}{}
\newenvironment{backward}{($\limplies$)}{}

% % Sets
\DeclareMathOperator{\supp}{supp}
\newcommand{\set}[1]{\left\{#1\right\}}
\newcommand{\st}{\text{ s.t. }}
\newcommand{\minus}{\setminus}
\newcommand{\subs}{\subseteq}
\newcommand{\ssubs}{\subsetneq}
\newcommand{\sups}{\supseteq}
\newcommand{\ssups}{\supset}
\DeclareMathOperator{\im}{Im}
\newcommand{\nothing}{\varnothing}
\DeclareMathOperator{\join}{\sqcup}
\DeclareMathOperator{\meet}{\sqcap}

% % Greek 
\newcommand{\al}{\alpha}
\newcommand{\be}{\beta}
\newcommand{\ga}{\gamma}
\newcommand{\de}{\delta}
\newcommand{\ep}{\varepsilon}
\newcommand{\ph}{\varphi}
\newcommand{\io}{\iota}
\newcommand{\ka}{\kappa}
\newcommand{\la}{\lambda}
\newcommand{\om}{\omega}
\newcommand{\si}{\sigma}

\newcommand{\Ga}{\Gamma}
\newcommand{\De}{\Delta}
\newcommand{\Th}{\Theta}
\newcommand{\La}{\Lambda}
\newcommand{\Si}{\Sigma}
\newcommand{\Om}{\Omega}

% % Mathbb
\newcommand{\bA}{\mathbb{A}}
\newcommand{\bB}{\mathbb{B}}
\newcommand{\bC}{\mathbb{C}}
\newcommand{\bD}{\mathbb{D}}
\newcommand{\bE}{\mathbb{E}}
\newcommand{\bF}{\mathbb{F}}
\newcommand{\bG}{\mathbb{G}}
\newcommand{\bH}{\mathbb{H}}
\newcommand{\bI}{\mathbb{I}}
\newcommand{\bJ}{\mathbb{J}}
\newcommand{\bK}{\mathbb{K}}
\newcommand{\bL}{\mathbb{L}}
\newcommand{\bM}{\mathbb{M}}
\newcommand{\bN}{\mathbb{N}}
\newcommand{\bO}{\mathbb{O}}
\newcommand{\bP}{\mathbb{P}}
\newcommand{\bQ}{\mathbb{Q}}
\newcommand{\bR}{\mathbb{R}}
\newcommand{\bS}{\mathbb{S}}
\newcommand{\bT}{\mathbb{T}}
\newcommand{\bU}{\mathbb{U}}
\newcommand{\bV}{\mathbb{V}}
\newcommand{\bW}{\mathbb{W}}
\newcommand{\bX}{\mathbb{X}}
\newcommand{\bY}{\mathbb{Y}}
\newcommand{\bZ}{\mathbb{Z}}
\newcommand{\id}{\mathbbm{1}}

% % Mathcal
\newcommand{\cA}{\mathcal{A}}
\newcommand{\cB}{\mathcal{B}}
\newcommand{\cC}{\mathcal{C}}
\newcommand{\cD}{\mathcal{D}}
\newcommand{\cE}{\mathcal{E}}
\newcommand{\cF}{\mathcal{F}}
\newcommand{\cG}{\mathcal{G}}
\newcommand{\cH}{\mathcal{H}}
\newcommand{\cI}{\mathcal{I}}
\newcommand{\cJ}{\mathcal{J}}
\newcommand{\cK}{\mathcal{K}}
\newcommand{\cL}{\mathcal{L}}
\newcommand{\cM}{\mathcal{M}}
\newcommand{\cN}{\mathcal{N}}
\newcommand{\cO}{\mathcal{O}}
\newcommand{\cP}{\mathcal{P}}
\newcommand{\cQ}{\mathcal{Q}}
\newcommand{\cR}{\mathcal{R}}
\newcommand{\cS}{\mathcal{S}}
\newcommand{\cT}{\mathcal{T}}
\newcommand{\cU}{\mathcal{U}}
\newcommand{\cV}{\mathcal{V}}
\newcommand{\cW}{\mathcal{W}}
\newcommand{\cX}{\mathcal{X}}
\newcommand{\cY}{\mathcal{Y}}
\newcommand{\cZ}{\mathcal{Z}}

% % Mathfrak
\newcommand{\f}[1]{\mathfrak{#1}}

% % Mathrsfs
\newcommand{\s}[1]{\mathscr{#1}}

% % Category Theory
\DeclareMathOperator{\obj}{Obj}
\DeclareMathOperator{\END}{End}
\DeclareMathOperator{\AUT}{Aut}
\newcommand{\CAT}{\mathrm{Cat}}
\newcommand{\SET}{\mathrm{Set}}
\newcommand{\TOP}{\mathrm{Top}}
\newcommand{\MON}{\mathrm{Mon}}
\newcommand{\GRP}{\mathrm{Grp}}
\newcommand{\AB}{\mathrm{Ab}}
\newcommand{\RING}{\mathrm{Ring}}
\newcommand{\CRING}{\mathrm{CRing}}
\newcommand{\MOD}{\mathrm{Mod}}
\newcommand{\VEC}{\mathrm{Vec}}
\newcommand{\ALG}{\mathrm{Alg}}
\newcommand{\ORD}{\mathrm{Ord}}
\newcommand{\POSET}{\mathbf{PoSet}}
\newcommand{\map}[2]{\yrightarrow[#2][2.5pt]{#1}[-1pt]}
\newcommand{\iso}[1][]{\cong_{#1}}
\newcommand{\OP}{\mathrm{op}}
\newcommand{\darrow}{\downarrow}
\newcommand{\LIM}{\varprojlim}
\newcommand{\COLIM}{\varinjlim}
\DeclareMathOperator{\coker}{coker}
\newcommand{\fall}[2]{\downarrow_{#2}^{#1}}
\newcommand{\lift}[2]{\uparrow_{#1}^{#2}}

% % Algebra
\newcommand{\nsub}{\trianglelefteq}
\newcommand{\inv}{{-1}}
\newcommand{\dvd}{\,|\,}
\DeclareMathOperator{\ev}{ev}

% % Analysis
\newcommand{\abs}[1]{\left\vert #1 \right\vert}
\newcommand{\norm}[1]{\left\Vert #1 \right\Vert}
\renewcommand{\bar}[1]{\overline{#1}}
\newcommand{\<}{\langle}
\renewcommand{\>}{\rangle}
\renewcommand{\hat}[1]{\widehat{#1}}
\renewcommand{\check}[1]{\widecheck{#1}}
\newcommand{\dsum}[2]{\sum_{#1}^{#2}}
\newcommand{\dprod}[2]{\prod_{#1}^{#2}}
\newcommand{\del}[2]{\frac{\partial#1}{\partial#2}}
\newcommand{\res}[2]{{% we make the whole thing an ordinary symbol
  \left.\kern-\nulldelimiterspace % automatically resize the bar with \right
  #1 % the function
  %\vphantom{\big|} % pretend it's a little taller at normal size
  \right|_{#2} % this is the delimiter
  }}

% % Galois
\DeclareMathOperator{\GAL}{Gal}
\DeclareMathOperator{\ORB}{Orb}
\DeclareMathOperator{\STAB}{Stab}
\newcommand{\emb}[3]{\mathrm{Emb}_{#1}(#2, #3)}
\newcommand{\Char}[1]{\mathrm{Char}#1}

%% code from mathabx.sty and mathabx.dcl to get some symbols from mathabx
\DeclareFontFamily{U}{mathx}{\hyphenchar\font45}
\DeclareFontShape{U}{mathx}{m}{n}{
      <5> <6> <7> <8> <9> <10>
      <10.95> <12> <14.4> <17.28> <20.74> <24.88>
      mathx10
      }{}
\DeclareSymbolFont{mathx}{U}{mathx}{m}{n}
\DeclareFontSubstitution{U}{mathx}{m}{n}
\DeclareMathAccent{\widecheck}{0}{mathx}{"71}

% Arrows with text above and below with adjustable displacement
% (Stolen from Stackexchange)
\newcommandx{\yaHelper}[2][1=\empty]{
\ifthenelse{\equal{#1}{\empty}}
  % no offset
  { \ensuremath{ \scriptstyle{ #2 } } } 
  % with offset
  { \raisebox{ #1 }[0pt][0pt]{ \ensuremath{ \scriptstyle{ #2 } } } }  
}

\newcommandx{\yrightarrow}[4][1=\empty, 2=\empty, 4=\empty, usedefault=@]{
  \ifthenelse{\equal{#2}{\empty}}
  % there's no text below
  { \xrightarrow{ \protect{ \yaHelper[ #4 ]{ #3 } } } } 
  % there's text below
  {
    \xrightarrow[ \protect{ \yaHelper[ #2 ]{ #1 } } ]
    { \protect{ \yaHelper[ #4 ]{ #3 } } } 
  } 
}

% xcolor
\definecolor{darkgrey}{gray}{0.10}
\definecolor{lightgrey}{gray}{0.30}
\definecolor{slightgrey}{gray}{0.80}
\definecolor{softblue}{RGB}{30,100,200}

% hyperref
\hypersetup{
      colorlinks = true,
      linkcolor = {softblue},
      citecolor = {blue}
}

\newcommand{\link}[1]{\hypertarget{#1}{}}
\newcommand{\linkto}[2]{\hyperlink{#1}{#2}}

% Perpage
\MakePerPage{footnote}

% Theorems

% % custom theoremstyles
\newtheoremstyle{definitionstyle}
{5pt}% above thm
{0pt}% below thm
{}% body font
{}% space to indent
{\bf}% head font
{\vspace{1mm}}% punctuation between head and body
{\newline}% space after head
{\thmname{#1}\thmnote{\,\,--\,\,#3}}

\newtheoremstyle{exercisestyle}%
{5pt}% above thm
{0pt}% below thm
{\it}% body font
{}% space to indent
{\it}% head font
{.}% punctuation between head and body
{ }% space after head
{\thmname{#1}\thmnote{ (#3)}}

\newtheoremstyle{examplestyle}%
{5pt}% above thm
{0pt}% below thm
{\it}% body font
{}% space to indent
{\it}% head font
{.}% punctuation between head and body
{\newline}% space after head
{\thmname{#1}\thmnote{ (#3)}}

\newtheoremstyle{remarkstyle}%
{5pt}% above thm
{0pt}% below thm
{}% body font
{}% space to indent
{\it}% head font
{.}% punctuation between head and body
{ }% space after head
{\thmname{#1}\thmnote{\,\,--\,\,#3}}

\newtheoremstyle{questionstyle}%
{5pt}% above thm
{0pt}% below thm
{}% body font
{}% space to indent
{\it}% head font
{?}% punctuation between head and body
{ }% space after head
{\thmname{#1}\thmnote{\,\,--\,\,#3}}

% Custom Environments

% % Theorem environments

\theoremstyle{definitionstyle}
\newmdtheoremenv[
    linewidth = 2pt,
    leftmargin = 0pt,
    rightmargin = 0pt,
    linecolor = darkgrey,
    topline = false,
    bottomline = false,
    rightline = false,
    footnoteinside = true
]{dfn}{Definition}
\newmdtheoremenv[
    linewidth = 2 pt,
    leftmargin = 0pt,
    rightmargin = 0pt,
    linecolor = darkgrey,
    topline = false,
    bottomline = false,
    rightline = false,
    footnoteinside = true
]{prop}{Proposition}
\newmdtheoremenv[
    linewidth = 2 pt,
    leftmargin = 0pt,
    rightmargin = 0pt,
    linecolor = darkgrey,
    topline = false,
    bottomline = false,
    rightline = false,
    footnoteinside = true
]{cor}{Corollary}

\theoremstyle{exercisestyle}
\newmdtheoremenv[
    linewidth = 0.7 pt,
    leftmargin = 20pt,
    rightmargin = 0pt,
    linecolor = darkgrey,
    topline = false,
    bottomline = false,
    rightline = false,
    footnoteinside = true
]{ex}{Exercise}
\newmdtheoremenv[
    linewidth = 0.7 pt,
    leftmargin = 20pt,
    rightmargin = 0pt,
    linecolor = darkgrey,
    topline = false,
    bottomline = false,
    rightline = false,
    footnoteinside = true
]{lem}{Lemma}

\theoremstyle{examplestyle}
\newmdtheoremenv[
    linewidth = 0.7 pt,
    leftmargin = 0pt,
    rightmargin = 0pt,
    linecolor = darkgrey,
    topline = false,
    bottomline = false,
    rightline = false,
    footnoteinside = true
]{eg}{Example}
\newmdtheoremenv[
    linewidth = 0.7 pt,
    leftmargin = 0pt,
    rightmargin = 0pt,
    linecolor = darkgrey,
    topline = false,
    bottomline = false,
    rightline = false,
    footnoteinside = true
]{ceg}{Counter Example}

\theoremstyle{remarkstyle}
\newtheorem{rmk}{Remark}

\theoremstyle{questionstyle}
\newtheorem{question}{Question}

\newenvironment{proof1}{
  \begin{proof}\renewcommand\qedsymbol{$\blacksquare$}
}{
  \end{proof}
} % Proofs ending with black qedsymbol 

% % tikzcd diagram 
\newenvironment{cd}{
    \begin{figure}[H]
    \centering
    \begin{tikzcd}
}{
    \end{tikzcd}
    \end{figure}
}

% tikzcd
% % Substituting symbols for arrows in tikz comm-diagrams.
\tikzset{
  symbol/.style={
    draw=none,
    every to/.append style={
      edge node={node [sloped, allow upside down, auto=false]{$#1$}}}
  }
}

\addbibresource{mybib.bib}

\begin{document}

\title{The adic Fargues--Fontaine curve}

\author{Ken Lee}
\date{February 2025}
\maketitle

\begin{abstract}
  These are expanded notes for a talk on the adic Fargues--Fontaine curve.
\end{abstract}

\tableofcontents

\section{Motivation : geometrization of $\ell$-adic local Langlands}

This section is largely informal.
We begin with local class field theory.

\begin{eg}
  Let $\bQ_p^{\mathrm{ab}} / \bQ_p$ be the maximal abelian extension of $\bQ_p$.
  The local Kronecker--Weber theorem says
  $\bQ_p^\mathrm{ab} = \bQ_p(\mu_{\bN}) = 
  \bQ_p(\mu_{p^\infty}) \breve{\bQ}_p$ where
  $\breve{\bQ}_p = \bQ_p(\bigcup_{(N , p) = 1} \mu_N)$
  is the maximal unramified extension of $\bQ_p$.
  Then we have : 
  \begin{cd}
    1 & {\mathrm{Gal}(\mathbb{Q}_p^\mathrm{ab} / \mathbb{Q}_p^\mathrm{ur})} & {\mathrm{Gal}(\mathbb{Q}_p^\mathrm{ab} / \mathbb{Q}_p)} & {\mathrm{Gal}(\overline{\mathbb{F}_p} / \mathbb{F}_p)} & 1 \\
    1 & {\mathrm{Gal}(\mathbb{Q}_p(\mu_{p^\infty}) / \mathbb{Q}_p)} & {W^{\mathrm{ab}}} & {\mathrm{Frob}^{\mathbb{Z}}} & 1 \\
    1 & {\mathbb{Z}_p^\times} & {\mathbb{Q}_p^\times} & {p^\mathbb{Z}} & 1
    \arrow[from=1-1, to=1-2]
    \arrow[from=1-2, to=1-3]
    \arrow["\simeq"{description}, draw=none, from=1-2, to=2-2]
    \arrow[from=1-3, to=1-4]
    \arrow[from=1-4, to=1-5]
    \arrow[hook', from=2-3, to=1-3]
    \arrow[hook', from=2-4, to=1-4]
    \arrow[from=3-1, to=3-2]
    \arrow["\simeq"{description}, draw=none, from=3-2, to=2-2]
    \arrow[from=3-2, to=3-3]
    \arrow["\simeq"{description}, draw=none, from=3-3, to=2-3]
    \arrow[from=3-3, to=3-4]
    \arrow["\simeq"{description}, draw=none, from=3-4, to=2-4]
    \arrow[from=3-4, to=3-5]
    \arrow[from=2-1, to=2-2]
    \arrow[from=2-2, to=2-3]
    \arrow[from=2-3, to=2-4]
    \arrow[from=2-4, to=2-5]
  \end{cd}
  Local class field theory is the isomorphism of the middle row
  to the bottom row.
  The isomorphism $\bQ_p^\times \simeq W^\mathrm{ab}$
  is the local Artin reciprocity map.
\end{eg}
The above can be seen as saying 
we have a correspondence
\begin{align*}
  \text{1-dim reps of $\GL_1(\bQ_p)$ } \leftrightarrow 
  \text{ 1-dim reps of $W$}
\end{align*}
where $W \subs \GAL(\overline{\bQ_p} / \bQ_p)$ 
is the preimage of $\FROB^\bZ \subs \GAL(\overline{\bF_p} / \bF_p)$.
More generally, $\ell$-adic local Langlands for $\GL_n$ and $\bQ_p$
says there is a ``nice'' correspondence 
\begin{align*}
  \text{$\ell$-adic reps of $\GL_n(\bQ_p)$} \leftrightarrow 
  \text{ group morphisms $W \to \GL_n(\overline{\bQ}_\ell)$}
\end{align*}
More generally still, 
$\ell$-adic local Langlands for 
a connected reductive group $G$ over $\bQ_p$ 
is about understanding the relation between \[
  \text{$\ell$-adic reps of $G(\bQ_p)$ }
  \overset{?}{\leftrightsquigarrow}
  \text{continuous group sections 
    $W \to \widehat{G}(\overline{\bQ}_\ell) \rtimes W$}
\]
where \begin{enumerate}
  \item $\widehat{G}$ is the connected reductive group over $\bZ_\ell$
  with root datum dual to $G \times_{\bQ_p} \overline{\bQ_p}$.
  \item The Weil group $W$ acts on the root datum of 
  $G \times_{\bQ_p} \overline{\bQ_p}$
  and hence on $\widehat{G}$ and its $\overline{\bQ}_\ell$-points.
  The semi-direct product $\widehat{G}(\overline{\bQ}_\ell) \rtimes W$
  is formed using this action.
\end{enumerate}
Objects on the right are called \emph{$L$-parameters}.
\cite[Def. VIII.1.1.]{FS24}
We have the following questions : 
\begin{enumerate}
  \item [Q1] How does the dual group $\widehat{G}$ appear?
  \item [Q2] What does $W$ have to do with the representation theory of 
  $G(\bQ_p)$?
\end{enumerate}
The heuristic of \cite{FS24} is as follows : 
\begin{enumerate}
  \item [Step 0] In the setting of 
  geometric Langlands of function fields
  of smooth projective curves $X$ over either finite fields or $\bC$,
  the above two questions are understood better.
  \begin{enumerate}
    \item [A1] $\widehat{G}$ comes from the study of 
    modifications of $G$-bundles locally around a point,
    i.e. geometric Satake.
    \item [A2] This comes from excursion operators of \cite{Laf18}.
  \end{enumerate}
  \item [Step 1] 
  The map $\SPEC \bC_p \to (\SPEC \breve{\bQ}_p) / \varphi^\bZ$
  is a $W$-torsor so at least in the $\GL_n$ case,
  we should think about $L$-parameters as $\ell$-adic
  rank $n$ local systems on $(\SPEC \breve{\bQ}_p) / \varphi^\bZ$.
  This raises the idea of ``unramified geometric Langlands
  over $(\SPEC \breve{\bQ}_p) / \varphi^\bZ$''.
  
  \item [Step 2]
  For any hope of geometric Satake and excursion operators 
  over $(\SPEC \breve{\bQ}_p) / \varphi^\bZ$,
  we need a moduli of $G$-torsors over 
  $(\SPEC \breve{\bQ}_p) / \varphi^\bZ$.
  Observe that \[
    (\SPEC \breve{\bQ}_p) / \varphi^\bZ = 
    (\SPEC W(\overline{\bF_p})[1 / p]) / \varphi^\bZ
  \]
  where $W(\_)$ is the functor of Witt vectors on perfect $\bF_p$-algebras 
  and $\varphi$ comes from lifting the Frobenius
  automorphism of $\overline{\bF_p}$.
  The idea of \cite{FS24} is to 
  think of the above as 
  \[
    \text{``$(\SPEC \overline{\bF_p} \times \SPEC \bQ_p )
    / (\varphi^\bZ \times \id)$''}
  \]
  and replace $\overline{\bF_p}$ with perfectoids $S$ over $\bF_p$.
  We thus want to consider : 
  \[
    X_S := \text{``$(S \times \SPEC \bQ_p)
    / (\varphi^\bZ \times \id)$''}
  \]
\end{enumerate}
The formula for $\breve{\bQ}_p$ suggests the following definition.
\begin{dfn}

  For $S = \SPA (C , \cO_C)$ where 
  $C$ is a perfectoid field of characteristic $p$,
  \[
    X_C := (\SPA W(\cO_C) \setminus V([\pi] p)) / \varphi^\bZ
  \]
  where $\pi \in \cO_C$ is a pseudo-uniformizer of $C$.
  This is the \emph{adic Fargues--Fontaine curve over $C$}.
  More generally,
  for $S = \SPA(A , A^+)$ an affinoid perfectoid in characteristic $p$,
  \[
    X_S := (\SPA W(A^+) \setminus V([\pi] p)) / \varphi^\bZ
  \]
  where $\pi \in A^+$ is a pseudo-uniformizer of $A$.
  This is the \emph{relative Fargues--Fontaine curve $X_S$ over $S$}.
\end{dfn}
The product $S \times \SPEC \bQ_p$ gives the empty scheme,
and similarly for $S \times_{\SPA \bZ} \SPA(\bQ_p , \bZ_p)$.
But one can show that the associated diamond of $X_S$
can be identified with
\[
  X_S^\diamond = (S \times \SPD \bQ_p) / (\varphi^\bZ \times \id)
\]

\textbf{Warning} : 
Since $(W(A^+) , W(A^+))$ is not discrete,
nor finite type over a Noetherian ring, nor strongly Noetherian,
nor sousperfectoid,
it is a priori not even clear that we have an adic space
before modding by the action of $\varphi$.
We make this precise in the section on adic spaces as functors.

\section{Holomorphic functions in the variable $p$}

Before anything, we give some intuition for the the case over a point.
First, a reminder of how the tilting equivalence works,
with a few extra details.
% For this, I find the theory of Witt vectors helpful.
% The key is the following lemma : 
% \begin{lem}
%   Let $A , B$ be rings
%   and $\bar{f} : A / p A \to B / p B$ a morphism of rings.
%   Assume $A / p A$ is \emph{perfect},
%   i.e. the Frobenius map $x \mapsto x^p$ is an isomorphism.
%   Then there is a unique multiplicative map
%   $\bar{f}_n$ such that the diagram of multiplicative monoids commute
%   \begin{cd}
%     & {B / p^{n+1} B} \\
%     {A / pA} & {B / pB}
%     \arrow[from=1-2, to=2-2]
%     \arrow["{\bar{f}_n}", dashed, from=2-1, to=1-2]
%     \arrow["{\bar{f}}"', from=2-1, to=2-2]
%   \end{cd}
%   \begin{proof1}(Skip in talk)
%     We first define $\bar{f}_n$.
%     Let $x \in A / p A$.
%     For lifts $y_n , \tilde{y}_n \in B$ of $\bar{f}(x)$,
%     we do not neccesarily have $y_n = \tilde{y}_n \in B / p^{n+1} B$.
%     However by the binomial theorem we do have the following observation : 
%     \begin{align}
%       y_n = \tilde{y}_n \in B / pB 
%       \implies y_n^{p^n} = \tilde{y}_n^{p^n} \in B / p^{n+1} B
%     \end{align}
%     Inspired by this observation, one proceeds as follows : 
%     \begin{enumerate}
%       \item Because $A / p A$ is perfect,
%       there exists a unique $x^{1 / p^n} \in A / pA$ such that
%       $(x^{1 / p^n})^{p^n} = x$.
%       \item Let $y_n \in$
%       be \emph{any} lift of $\bar{f}(x^{1 / p^n})$.
%       \item Define $\bar{f}_n(x) := (y_n)^{p^n}$ in $B / p^{n+1} B$.
%       By observation (1),
%       this only depends $y_n$ mod $p$ which is $\bar{f}(x^{1 / p^n})$.
%       In particular $\bar{f}_n$ is unique.
%       \item Multiplicativity is a straightforward calculation.
%     \end{enumerate}
%   \end{proof1}
% \end{lem}
% If $A$ has perfect reduction mod $p$ and is $p$-adically complete,
% we can apply the above lemma to the identity of $A / p A$
% to obtain a multiplicative section \[
%   [\_] : A / p A \to A
% \]
% This is called the \emph{Teichmuller lift}.
% We can then expand elements of $A$ as power series in $p$ with
% coefficients in Teichmuller lifts.
% One can use this to lift morphisms on mod $p$ reductions.
% \begin{prop}[Theory of Witt vectors]
%   Define a \emph{strict $p$-ring} as a ring which is 
%   \begin{enumerate}
%     \item (Existence of Teichmuller expansion) 
%     $p$-adically complete, perfect modulo $p$
%     \item (Uniqueness of Teichmuller expansion) $p$-torsion-free
%   \end{enumerate}
%   Then the following are true : 
%   \begin{enumerate}
%     \item Let $A$ be a strict $p$-ring
%     and $B$ $p$-adically complete.
%     For $\bar{f} : A / p A \to B / p B$ a ring morphism,
%     define $f : A \to B$ by
%     \begin{cd}
%       A & B & {} \\
%       {A / p A} & {B / p B} & {}
%       \arrow["f", from=1-1, to=1-2]
%       \arrow[from=1-1, to=2-1]
%       \arrow[from=1-2, to=2-2]
%       \arrow["{f^\sharp}"{description}, from=2-1, to=1-2]
%       \arrow["{\bar{f}}"', from=2-1, to=2-2]
%       \arrow["{f(x) = \sum_{n \geq 0} f^\sharp(\bar{x_n}) p^n}",
%         from=1-3, to=2-3, draw = none]
%     \end{cd}
%     where $x = \sum_{n \geq 0} [\bar{x_n}] p^n$ is the
%     unique expansion in terms of Teichmuller lifts
%     and $f^\sharp$ is the section obtained from the previous lemma.
%     Then $f$ is the unique ring morphism such that 
%     $\bar{f}$ mod $p$.
%     In particular, 
%     reduction mod $p$ is fully faithful on strict $p$-rings.
%     \item (Essentially surjective) 
%     For any perfect $\bF_p$-algebra $C$,
%     there exists a strict $p$-ring $W(C)$ such that
%     $W(C) / p W(C) \simeq C$.
%   \end{enumerate}
%   Thus, mod $p$ yields an equivalence between
%   the category of strict $p$-rings and the category of perfect $\bF_p$-algebras.

% \end{prop}
% A reminder of the following example of tilting for perfectoid fields :
% \begin{eg}
%   Let $K := \bQ_p^\infty := \bZ_p[p^{1 / p^\infty}]^\wedge_p[1 / p]$.
%   One can define \begin{align*}
%     \cO_K^\flat := \LIM_{\phi} \cO_K / (p)\simeq \bF_p[[t^{1 / p^\infty}]]^\wedge_t &&
%     K^\flat := \FRAC\, \cO_K^\flat \simeq \bF_p((t^{1 / p^\infty}))
%   \end{align*}
%   where $t = (p , p^{1 / p} , p^{1 / p^2} , \dots)$.
%   Furthermore, we can recover $K$ from $K^\flat$ via the theory of Witt vectors.
%   \[\begin{tikzcd}
% 	  {W(\mathbb{F}_p[[t^{1 / p^\infty}]]^\wedge_t)} 
%       & {\mathbb{Z}_p[p^{1 / p^\infty}]^\wedge_p} 
%       & {\theta(x) = 
%       \theta(\sum_{n \geq 0} [\bar{x_n}] p^n)
%       := \sum_{n \geq 0} \bar{x_n}^\sharp p^n}\\
% 	  {\mathbb{F}_p[[t^{1 / p^\infty}]]^\wedge_t} 
%       & {\mathbb{Z}_p[p^{1 / p^\infty}]^\wedge_p / (p)}
%       &
% 	  \arrow["\theta", from=1-1, to=1-2]
% 	  \arrow[from=1-1, to=2-1]
% 	  \arrow[from=1-2, to=2-2]
% 	  \arrow["{\_^\sharp}"{description}, from=2-1, to=1-2]
% 	  \arrow[from=2-1, to=2-2]
%     \end{tikzcd}\]
%   We can prove $\theta$ is surjective.
%   $\bZ_p[p^{1 / p^\infty}]^\wedge_p / (p)$ has surjective Frobenius
%   so $\bF_p[[t^{1 / p^\infty}]]^\wedge 
%   \to \bZ_p[p^{1 / p^\infty}]^\wedge_p / (p)$
%   is surjective.
%   Choosing a section gives 
%   a system of coefficients for $\bZ_p[p^{1 / p^\infty}]^\wedge_p$ 
%   factoring through $\theta$.
%   \[
%     \bZ_p[p^{1 / p^\infty}]^\wedge_p / (p) \map{}{}
%     \bF_p[[t^{1 / p^\infty}]]^\wedge_t \map{[\_]}{}
%     W(\bF_p[[t^{1 / p^\infty}]]^\wedge_t) \map{\theta}{}
%     \bZ_p[p^{1 / p^\infty}]^\wedge_p
%   \]
%   This implies $\theta$ is surjective.
%   We have \[
%     \theta([t]) = t^\sharp = \lim_{n \to \infty} (p^{1 / p^n})^{p^n}
%     = p
%   \]
%   so $p - [t] \in \ker \theta$. One can show $(p - [t]) = \ker \theta$.
% \end{eg}
% The general phenomenon is the following : 
\begin{prop}[Tilting correspondence / Classification of untilts]
  
  Define the following category : 
  \begin{itemize}
    \item objects are $(F , I)$
    where $F$ is a perfectoid field of characteristic $p$
    and $I \subs W(F^{\circ})$ which is generated by
    a \emph{primitive element of degree 1}.
    These are elements $x \in W(F^\circ)$
    with Teichmuller expansion \[
      x = [\bar{x}_0] + [\bar{x}_1] p^1 + \cdots
    \]
    where $\bar{x}_0 \in F^{\circ\circ}$ and
    $\bar{x}_1 \in (F^\circ)^\times$.\footnote{
      In \cite[Def. 2.2.1]{FF18},
      $\bar{x}_0$ is required to be non-zero.
      This has the effect of excluding the (unique up to isomorphism)
      characteristic $p$ untilt.
      The definition here argees with
      \cite[Def. 6.2.9]{SW20}.
    }
    
    \item a morphism $(F , I) \to (F_1 , I_1)$
    is a continuous ring morphism $F \to F_1$ which
    sends $I$ into $I_1$ under the corresponding 
    ring morphism $W(F) \to W(F_1)$.
  \end{itemize}
  We have an equivalence : 
  \begin{cd}
    \begin{matrix}
      \text{
        category of perfectoid fields $K$
      } \\
      \text{with continuous ring morphisms}
    \end{matrix}
    & \text{
      category of $(F , I)$
    }
    \arrow[from=1-1, to=1-2 , "{\sim}"]
    \arrow[from=1-2, to=1-1]
  \end{cd}
  The two functors on objects are \begin{enumerate}
    \item Given a perfectoid field $K$,
    take $(K^\flat , \ker \theta \subs W(K^{\flat , \circ}))$.
    The norm on $K^\flat$ can be defined through
    \[
      K^{\flat , \circ} \map{\_^\sharp}{} 
      K^\circ \map{\abs{\_}}{} \bR_{\geq 0}
    \]
    \item Given $(F , I)$, form
    $F^{\sharp , \circ} := W(F^\circ) / I$.
    For $x \in F^{\sharp , \circ}$
    there exists $\tilde{x} \in F^\circ$ 
    unique up to $(F^\circ)^\times$ such that
    $[\tilde{x}] = x$ in $F^{\sharp , \circ}$.
    Defining $\abs{x} := \abs{\tilde{x}}$ defines
    a norm on $F^{\sharp , \circ}$ making it a rank 1 valuation ring
    and $F^\sharp := \FRAC \, F^{\sharp , \circ}$ a perfectoid field.
  \end{enumerate}
  \cite[Theorem 1.5.1]{Ked15}
\end{prop}

\begin{dfn}[Radius of untilts]
  Fix a perfectoid field $F$ of characteristic $p$.
  An untilt of $F$ is defined
  as the data $(K , \io)$ where 
  \begin{itemize}
    \item $K$ is a perfectoid field
    \item $\io : K^\flat \simeq F$ is a bi-continuous ring isomorphism.
  \end{itemize}
  By the theory of Witt vectors, $\io$ corresponds
  to a morphism between Witt vectors : 
  \begin{cd}
    {W(K^{\flat , \circ})} & {K^\circ} \\
    {W(F^\circ)} & {W(F^\circ) / (p - [\iota(p^\flat)])}
    \arrow["\theta", two heads, from=1-1, to=1-2]
    \arrow["\sim"', from=1-1, to=2-1]
    \arrow["\sim", from=1-2, to=2-2]
    \arrow[two heads, from=2-1, to=2-2]
  \end{cd}
  which yields a bi-continuous ring isomorphism
  $K \simeq \FRAC\, W(F^\circ) / (p - [\io(p^\flat)])$.
  Here $p^\flat \in K^\flat$ is either
  \begin{enumerate}
    \item zero when $K$ is characteristic $p$
    \item a pseudo-uniformizer coming from a choice of
    perfectoid pseudo-uniformizer in $K$.
  \end{enumerate}
  In either cases, the norm of $p^\flat$ is independent of choices,
  and so we have a well-defined number \[
    r(K , \io) := \abs{\io(p^\flat)}_F \in [0 , 1)
  \]
  This is called the \emph{radius}.
  A morphism of untilts $(K_1 , \io_1) \to (K_2 , \io_2)$ is
  a continuous ring morphism $\al : K_1 \to K_2$
  such that $\io_2 = \al^\flat \io_1$.
\end{dfn}
Isomorphic untilts gives rise to the same radius
and up to isomorphism $F$ is the only untilt of itself with radius zero.
This gives rise to the following heuristic :
\begin{align*}
  \abs{\cY_F} := \set{\text{Untilts of $F$}}/\simeq 
  && \set{\abs{z} < 1} \subs \bC \\
  p \in W(F^\circ) && \text{ coordinate function $z$} \\
  K \mapsto \abs{p}_K && z \mapsto \abs{z} \\
  f \in W(F^\circ) && \sum_{n = 0}^\infty c_n z^n 
    \text{ s.t. $\abs{c_n}$ bounded} \\
  ??? && \text{ring of holomorphic functions on $\set{\abs{z} < 1} \subs \bC$}
\end{align*}
Intuitively,
a holomorphic function on $\set{\abs{z} < 1}$
is a compatible family of holomorphic functions on 
$\set{a \leq \abs{z} \leq b}$ ranging over $[a , b] \subs [0 , 1)$.
Making rigorous the rings of ``holomorphic functions in variable $p$
on $a \leq \abs{p} \leq b$'' will give us an analytic space over $\bZ_p$
playing the role of $\set{\abs{z} < 1} \subs \bC$.
For the purposes of defining the Fargues--Fontaine curve,
we will jump forward to describing the holomorphic functions 
on ``$0 < \abs{p} < 1$''.

\section{Defining analytic adic spaces without using adic spectrum}

I want to define analytic adic spaces without using adic spectrum
because : 
\begin{enumerate}
  \item As mentioned before, $W(R^+)$ is not sheafy so
  $\SPA W(R^+)$ already falls out of 
  category (V). (For the definition, see \cite[Def. 3.2.1]{SW20}.)
  \item \cite[Prop. 11.2.1]{SW20} does not explicitly define
  $\cY_S := \SPA W(R^+) \setminus V([\pi])$ 
  as an object of category (V).
  The underlying topological space is conceivable,
  however defining a structure presheaf and showing it is a sheaf
  looks like non-trivial work.
  \item Verifying the sheaf condition on structure presheaves
  requires an understanding of the underlying topological space.
  I find adic spectra difficult to compute.
\end{enumerate}
In the end,
no matter what theory of adic spaces one chooses,
it should always be the case that a filtered colimit of
sheafy Tate affinoids along inclusions of rational domains gives an adic space.
The relative Fargues--Fontaine curve $X_S$ will be such a colimit.
We develop adic geometry completely analogously to 
algebraic geometry using functor of points in the following steps : 
\begin{enumerate}
  \item We define Tate affinoids as
  dual to sheafy complete Tate--Huber pairs
  and introduce the Grothendieck topology given by 
  standard rational covers.
  \item
  Define analytic adic spaces as living in the sheaf topos over Tate affinoids
  in the same way as schemes can be defined for algebraic geometry.
\end{enumerate}
This is more restrictive than 
the theory of pre-adic spaces in \cite[Section 3.4]{SW20}.
We chose to do restrict to sheafy complete Tate--Huber pairs because 
\begin{enumerate}
  \item Aesthetic reason : we think sheafifying representables is too opaque;
  every reasonable site structure should be subcanonical.
  \item The category of complete Huber pairs
  does not admit pushouts.
  This is problematic for considering site structures.
  \item One may object that 
  we do not need all fiber products of representables : 
  for a site structure we only need to know arrows in covers
  are closed under pullback along any morphism of representables.
  However, it is unclear whether pushouts along rational localizations
  yield rational localizations for general
  morphisms between complete Huber pairs.
\end{enumerate}
The drawbacks this approach are : 
\begin{enumerate}
  \item We miss out on schemes and formal schemes,
  say over $\SPA(\bZ_p , \bZ_p)$.
  \item We have to keep things like ``$\SPA(W(\bZ_p) , W(\bZ_p))$''
\end{enumerate}
A different fix, which requires more machinary,
is to realise that
every complete Huber pair is sheafy in a derived sense,
e.g. in the Clausen--Scholze theory of analytic stacks.
\begin{dfn}
  
  Define $\widetilde{\AFF}$ as
  opposite to the category of complete Huber pairs
  with continuous ring morphisms preserving subrings of integral elements.
  For $(A , A^+)$ complete Huber,
  we write $\widetilde{\SPA}(A , A^+)$ for the corresponding object in
  $\widetilde{\AFF}$.

  Define $\AFF_\TATE \subs \widetilde{\AFF}$ as the full subcategory
  dual to sheafy complete Tate--Huber pairs.
  We write $\SPA (A , A^+)$ similarly.

  We identify $\widetilde{\AFF} , \AFF_\TATE$ with
  their essential images in presheaves under the Yoneda embedding.
  Objects in $\AFF_\TATE$ we call \emph{Tate affinoids}.
\end{dfn}

\begin{prop}
  
  Let $\al : (R , R^+) \to (A , A^+), \be : (R , R^+) \to (B , B^+)$ 
  be two morphisms 
  of complete Huber pairs which are \emph{adic}.
  This implies we can choose
  rings of definitions $R_0 , A_0 , B_0$ for $R , A , B$
  and an ideal $I \subs R_0$ of definition such that
  $\al(I) A_0 \subs A_0$ and $\be(I) B_0 \subs B_0$ are ideals of definitions.
  Define $C := A \otimes_R B$
  and let $C_0 \subs C$ be the image of $ A_0 \otimes_{R_0} B_0$.
  Then \begin{enumerate}
    \item $C$ becomes a Huber ring by declaring $I C_0 \subs C_0$
    to be ideal and ring of definition.
    \item Let $C^+ \subs C$ be the integral closure of 
    the image of $A^+ \otimes_{R^+} B^+$ in $C$.
    Then $(C , C^+)$ is a Huber pair
    whose completion gives the pushout
    in the category of complete Huber pairs.
    \footnote{
      Before completion, it is in fact the pushout in
      the category of Huber pairs but we do not care for non-complete
      Huber pairs.
    }
  \end{enumerate}
  Since every morphism of complete Tate--Huber rings are adic,
  $\AFF_\TATE$ has fiber products.
  \cite[Lem. 5.1.2]{SW20}
\end{prop}

The following example shows 
that $\widetilde{\AFF}$ does not admit fiber products.

\begin{eg}[$\widetilde{\AFF}$ does not have fiber products]

  Consider $\widetilde{\SPA}(\bZ[T] , \bZ) \to \widetilde{\SPA}(\bZ , \bZ)
  \leftarrow \SPA(\bQ_p , \bZ_p)$.
  We can always take the fiber product in
  $\PSH \widetilde{\AFF}$.
  The inclusion $\widetilde{\AFF} \to \PSH \widetilde{\AFF}$
  preserves limits so 
  if the fiber product exists in $\widetilde{\AFF}$
  then it must agree with the one in the presheaf category.
  Here, $\widetilde{\SPA}(\bZ[T] , \bZ)$ 
  is the functor $(A , A^+) \mapsto A$,
  in other words $\bA^1$. 
  So we are trying to compute the affine line over $\SPA(\bQ_p , \bZ_p)$.
  We will see later that this is \emph{not} Tate affinoid.
\end{eg}

A reminder for the procedure to compute rational localizations.
This works for $\widetilde{\AFF}$.
\begin{prop}[Rational domains]
  Let $(R , R^+)$ be a complete Huber pair.
  Let $s \in R$ and $T \subs R$ finite subset with $TR \subs R$ open.
  Consider the subfunctor 
  $\set{\abs{t} \leq \abs{s} \neq 0 \st t \in T} \subs \widetilde{\SPA}(R , R^+)$
  defined by \[
    \set{\abs{t} \leq \abs{s} \neq 0 \st t \in T}(A , A^+) := \set{
      \text{ $\al \in \widetilde{\SPA}(R , R^+)(A , A^+) 
      \st s \in A^\times$ and
      $T / s \subs A^+$
      }
    }
  \]
  Then this is representable by $(R , R^+) \to (R\<T / s\> , R\<T / s\>^+)$
  defined as 
  \begin{enumerate}
    \item Choose $I \subs R_0 \subs R$ with $R_0$ a ring of definition
    and $I$ finitely generated ideal of definition.
    \item Make $R[1 / s]$ into a topological ring by 
    using the ideal and ring of definition 
    $I R_0 [T / s] \subs R_0[T / s] \subs R[1 / s]$.
    \item Complete the Huber ring $R[1 / s]$ 
    which can be computed as \begin{align*}
      \widehat{R[1 / s]} := 
      \widehat{R[1/s]}_0 \otimes_{R_0[T / s]} R[1 / s] \\
      \widehat{R[1/s]}_0 := \brkt{\LIM_{n \geq 0} \frac{R_0[T / s]}{(I R_0[T / s])^n}} 
    \end{align*}
    where $\widehat{R[1/s]}_0$ and $I \widehat{R[1/s]}_0$ serves as
    a ring and ideal of definition.
    It can be shown that $\widehat{R[1/s]}_0$
    is precisely the topological closure of $R_0[T/s]$ inside 
    $\widehat{R[1 / s]}$.
    \item Define $R[1 / s]^+$ as the integral closure of
    $R^+[T / s]$ inside $R[1 / s]$.
    Finally, define the complete Huber pair $(R\<T / s\> , R\<T / s\>^+)$
    as \begin{align*}
      R\<T / s\> &:= \widehat{R[1 / s]} \\
      R\<T / s\>^+ &:= \text{topological closure of $R[1 / s]^+$ in $R\<T / s\>$}
    \end{align*}
  \end{enumerate}
  \cite[Theorem 3.1.3]{SW20}

  If $\al : (R , R^+) \to (A , A^+)$ is an \emph{adic} morphism
  of complete Huber pairs,
  then $\al(T) R \subs R$ is open.
  \cite[Prop. III.2.5.(iv)]{Mor19}
  Consequently, the pullback of rational domains remain representable
  for Tate affinoids.
\end{prop} 

\begin{dfn}[Analytic topology on $\AFF_\TATE$]
  
  A standard rational cover of $\SPA(A , A^+) \in \AFF_\TATE$ is
  a collection of morphisms $(U_t \to \SPA(A , A^+))_{t \in T}$
  where $T \subs A$ is a finite subset with $(T) = A$ and 
  \[
    U_t := \set{\abs{t^\prime} \leq \abs{t} \neq 0 \st t \neq t^\prime \in T}
  \]
  \cite[Def. IV.2.3.1]{Mor19}
  One can show standard rational covers
  gives a site structure on $\AFF_\TATE$.
  We call the associated Grothendieck topology
  the \emph{analytic topology} on $\AFF_\TATE$.
  It is subcanonical.
\end{dfn}

\begin{dfn}[Analytic adic spaces as functors on a subcanonical site]

  We say $j : U \to \SPA(R , R^+) \in \AFF_\TATE$ 
  is an \emph{open immersion with affinoid target} when
  it is a monomorphism of presheaves
  and there exists a collection $(U_i)_{i\in I}$ of rational opens
  of $\SPA(R , R^+)$ such that 
  \begin{enumerate}
    \item for all $i\in I$ we have $U_i \subs U$
    \item for all $\SPA (A , A^+) \to U$,
    there exists a standard rational cover $(W_j)_{j \in J}$ 
    of $\SPA(A , A^+)$ such that each $W_j \to \SPA(A , A^+) \to U$ 
    factors through some $U_i \to U$.
  \end{enumerate}

  We say $j : U \to X$ in $\PSH \AFF_\TATE$ is an \emph{open immersion}
  when the base change to all affinoids mapping into $X$ gives
  open immersions with affinoid target.

  An \emph{analytic adic space} 
  is a sheaf $X$ over $\AFF_\TATE$ for the analytic topology
  such that there exists a collection of open immersions $(U_i \to X)_{i \in I}$
  where \begin{enumerate}
    \item each $U_i$ is Tate affinoid
    \item for all $\SPA(A , A^+) \to X$,
    there exists a standard rational cover $(W_j)_{j \in J}$ of $\SPA(A , A^+)$
    such that each $W_j \to \SPA(A , A^+) \to X$ factors through some
    $U_i \to X$.
  \end{enumerate}
  Such a collection is called an atlas (for the analytic topology).
\end{dfn}
\begin{prop}
  
  Let $X : I \to \AFF_\TATE$ be a filtered system of Tate affinoids
  with transition maps which are rational localizations.
  Let $X := \COLIM_{i \in I} X_i$ be the colimit
  in the category of \emph{presheaves} over $\AFF_\TATE$.
  Then $X$ is in fact the colimit of the same diagram
  in $\SH \AFF_\TATE$ and is an adic space.
\end{prop}
\begin{proof}
  To show the presheaf colimit is already a sheaf,
  it follows from representables being quasi-compact.
  The key fact is that 
  every $\SPA (A , A^+) \to X$ admits 
  a factoring $\SPA(A , A^+) \to X_i \to X$.
  Then the fact that each $X_j \to X$ is an open immersion follows from
  the fact that rational localizations of affinoids
  are preserved under base change.
  The key fact also implies $(X_i)_{i \in I}$ form an atlas for $X$.
\end{proof}

Here is an example to demonstrating you do not need the adic spectrum.
\begin{eg}[Affine line over $\SPA(\bQ_p , \bZ_p)$ as an adic space]
  We work with $\AFF_\TATE / \SPA(\bQ_p , \bZ_p)$.
  Consider the functor \[
    \bA^1 : \SPA(A , A^+) \mapsto A
  \]
  We show that this is an adic space over $\SPA(\bQ_p , \bZ_p)$.
  Consider $p \bZ_p \subs \bZ_p \subs \bQ_p[T]$.
  We topologize $\bQ_p[t]$ using $(p^n \bZ_p)_{n \geq 0}$.
  The ring $\bZ_p$ is integrally closed in $\bQ_p[T]$ 
  so we have a complete Huber pair $(\bQ_p[T] , \bZ_p)$.
  \begin{lem}
    
    $\widetilde{\SPA}(\bQ_p[T] , \bZ_p)$ on $\AFF_\TATE / \SPA(\bQ_p , \bZ_p)$
    is isomorphic to $\bA^1$.
    \begin{proof1}
      Let $a \in (A , A^+)$ with $(\bQ_p , \bZ_p) \to (A , A^+)$.
      To suffices to check the algebra morphism $\bQ_p[T] \to A , T \mapsto a$
      is continuous.
      Pick a ring and ideal of definition $I \subs A_0 \subs A$.
      By continuity of $\bQ_p \to A$,
      there exists $n \geq 0$ with $p^n \bZ_p \subs \bQ_p$ landing in $I$.
      This implies $p^n \bZ_p \subs \bQ_p[T]$ also lands in $I$.
    \end{proof1}
  \end{lem}
  Unfortunately, we don't know a way for checking if 
  $(\bQ_p[T] , \bZ_p)$ is sheafy : 
  $\bQ_p[T]$ is not finite type over the ring of definition $\bZ_p$.
  But consider \[
    \bD := \set{\abs{T} \leq 1} : \SPA(A , A^+) \mapsto A^+
  \]
  \emph{Idea : we can write $\bA^1$ as union of larger and larger disks.}
  Let $\al : (\bQ_p , \bZ_p) \to (A , A^+)$ of complete Huber pairs.
  Since $A^+ \subs A^\circ$,
  there exists a ring and ideal of definition $I \subs A_0 \subs A$
  such that $a \in A_0$.
  \begin{enumerate}
    \item $\bZ_p \to A_0$ : 
    By continuity of $\bQ_p \to A$,
    there exists $n \geq 0$ with $\bZ_p / p^n \bZ_p \to A_0 / I$.
    Since $A_0$ is $I$-adically complete,
    this defines a continuous morphism $\bZ_p \to A_0$
    which is a priori different from the given map $\bZ_p \to A$.
    However, $\bZ \to \bZ_p \to A_0 \to A$ agrees with
    the given $\bZ \to \bZ_p \to A$.
    Since $\bZ \subs \bZ_p$ is dense, we obtain that 
    the two morphisms $\bZ_p \to A$ agree and hence $\bZ_p \to A_0$.
    \item We extend $\bZ_p \to A_0$ to $\bZ_p[T] \to A_0$ by $T \mapsto a$.
    Since $p^n \in \bZ_p[T]$ maps into $I$ we have $p^n \bZ_p[T]$
    maps into $I$.
    By $I$-adic completeness of $A_0$,
    this extends uniquely to a continuous morphism \[
      \bZ_p\< T \> := \LIM_{n \geq 0} \frac{\bZ_p[T]}{p^{n+1} \bZ_p[T]} \to A_0
    \]
    Since $\bZ_p \to A^+$ we have $\bZ_p[T]$ maps into $A^+$.
    By definition of $(A , A^+)$ being complete,
    $A^+$ is closed in $A$ under the $I$-adic topology.
    Given that $\bZ_p[T] \to \bZ_p\<T\>$ has dense image,
    it follows that $\bZ_p\<T \>$ is mapped into $A^+$.
    \item By assumption, $p \in A^\times$ 
     and $\bQ_p\<T\> := \bZ_p\< T \> \otimes_{\bZ_p[T]} \bQ_p[T]
    = \bZ_p\<T\> \otimes_{\bZ_p[T]} \brkt{\bZ_p[T][1 / p]}
    = \bZ_p\<T\>[1/p]$ so
    the continuous morphism $\bZ_p\<T \> \to A^+$ extends
    uniquely to $\bQ_p\<T \> \to A$.
    We have rediscovered the classical Tate algebra!
    \end{enumerate}
  We have $\bD \simeq \SPA(\bQ_p\<T\> , \bZ_p\<T\>)$.
  The fact that this is sheafy is 
  Tate's acyclicity theorem.
  \cite[Theorem 3.1.8.(3)]{SW20}
  If we instead wanted to do $\set{\abs{T} \leq \abs{1 / p^n}} := 
  \set{\abs{p^n T} \leq 1}$
  parameterizing $a \in A$ such that $p^n a \in A^+$,
  we would get $\SPA(\bQ_p\<p^n T\> , \bZ_p\<p^n T\>)$.
  Since $A = A^+[1 / p]$ we have 
  \[
    \bA^1 \simeq \bigcup_{n \geq 0} \SPA(\bQ_p\<p^nT\> , \bZ_p\<p^n T\>)
  \]
  where the latter is a filtered colimit of affinoids
  along rational localizations.
  Hence $\bA^1$ is an adic space as it should be.

  Just to be sure : $\bZ_p\< p T \>$ is computed by taking the inverse limit of
  \begin{align*}
    \bZ_p[p T] / p \bZ_p[p T] &\simeq \bF_p \\
    \bZ_p[p T] / p^2 \bZ_p [pT] &\simeq 
      (\bZ_p / p^2)[p T] \subs (\bZ_p / p^2)[T]\\
    \bZ_p[pT] / p^3 \bZ_p [pT] &\simeq 
      (\bZ_p / p^3)[p T] \subs (\bZ_p / p^3)[T]\\
    &\vdots
  \end{align*}
  we obtain an injective ring morphism $\bZ_p\< pT \> \to \bZ_p[[T]]$
  with image \[
    \set{\sum_{n = 0}^\infty a_n p^n T^n \st \lim_{n \to \infty} \abs{a_n} = 0}
    = \set{\sum_{n = 0}^\infty b_n T^n \st 
    \lim_{n \to \infty} \abs{p^{-n} b_n} \to 0}
  \]
\end{eg}

The above computation for $\set{\abs{T} \leq 1}$ generalises
over any Tate affinoid.
\begin{lem}
  
  Let $\SPA(R , R^+) \in \AFF_\TATE$.
  Define $\set{\abs{T} \leq 1}$ as a presheaf over 
  $\AFF_\TATE / \SPA(R , R^+)$
  sending $\SPA (A , A^+) \mapsto A^+$.
  Then there exists a complete Tate--Huber pair whose $\widetilde{\SPA}$
  is isomorphic to $\set{\abs{T} \leq 1}$.

  \begin{proof1}
    By \cite[Lem. 5.1.2]{SW20}
    the morphism $\al : (R , R^+) \to (A, A^+)$ is adic
    i.e. there exists $I \subs R_0 \subs R$ ideal and ring of definition
    and $A_0 \subs A$ ring of definition with 
    $\al R_0 \subs A_0$ and $(\al I) A_0 \subs A_0$
    giving an ideal of definition.
    If $A_0$ is complete w.r.t. some ideal of definition,
    then it is complete w.r.t. any ideal of definition.
    In particular $A_0$ is $\al I$-adically complete.
  
    Now let $f \in A^+$.
    There exists a ring of definition $A_0^\prime \subs A$ with 
    $f \in A_0^\prime$.
    There exists a ring of definition $A_0^{\prime\prime} \subs A$
    which contains both $A_0 , A_0^\prime$.
    Since $\al$ is adic, $(\al I) A_0^{\prime \prime} \subs A_0^{\prime\prime}$
    is an ideal of definition.
    So WLOG $f \in A_0$.
    
    We extend $R_0 \to A_0$ to $R_0[T] \to A_0$ by $T \mapsto f$.
    We have that $I R_0[T]$ maps into $\al I$.
    By $\al I$-adic completeness of $A_0$
    we have a unique extension of ring morphisms 
    \begin{cd}
      {R_0[T]} & {A_0} \\
      {R_0[T]^{\wedge}_{I R_0[T]}} &
      \arrow[from=1-1, to=1-2]
      \arrow[from=1-1, to=2-1]
      \arrow["{\exists !}"', dashed, from=2-1, to=1-2]
      \arrow["{:=}"{description}, draw=none, from=2-1, to=2-2]
    \end{cd}
    We have an explicit description of the adic completion 
    as an $R_0[T]$ algebra : 
    \begin{align*}
      R_0[T]^{\wedge}_{I R_0[T]} \simeq 
      R_0\< T \> := \set{\sum_{n = 0}^\infty a_n T^n \in R_0[[T]] \st 
      \lim_{n \to \infty} a_n = 0}
    \end{align*}
    This is continuous when we equip $R_0\<T\>$ with the
    $IR_0\< T\>$-adic topology.
    Since $R$ is Tate, there exists $\pi \in R^{\circ\circ} \cap R^\times$
    with $R = R_0[1 / \pi]$ and $\pi \in R_0$.
    The following ideal of a subring of a ring then defines a
    complete Huber ring : 
    \[
      I R_0 \< T \> \subs R_0 \< T \> \subs R_0\< T \>[1 / \pi]
    \]
    We have the same explicit description : 
    \begin{align*}
      R_0\< T \>[1 / \pi] 
      \simeq R\< T \> := \set{\sum_{n = 0}^\infty a_n T^n \in R[[T]]  \st 
      \lim_{n \to \infty} a_n = 0}
   \end{align*}
    The morphism $R_0\<T \> \to A_0$ extends
    uniquely to a morphism $R\< T \> \to A$ of complete Huber rings.
  
    So far, the ideal and ring of definition $I \subs R_0 \subs R$
    depended on $\al$.
    However, for any ideal of ring of definition 
    $\widetilde{I} \subs \widetilde{R_0} \subs R$
    and pseudo-uniformizer $\widetilde{\pi}$ of $R$
    with $\widetilde{R_0}[1 / \widetilde{\pi}] = R$
    and additionally $R_0 \subs \widetilde{R_0}$,
    we have an isomorphism of complete Huber rings
    \[
      R\< T \> \simeq 
      \widetilde{R_0}[T]^\wedge_{\widetilde{I_0}\widetilde{R_0}[T]}[1 / \widetilde{\pi}]
    \]
    Since the set of rings of definition is filtered,
    we obtain that the topology on $R\< T \>$ is independent
    of the choice of $I \subs R_0 \subs R$.
    
    Define $R\< T \>^+$ as the topological closure in $R\< T \>$
    of the integral closure of $A^+[T]$ in $R[T]$.
    Then the complete Huber pair $(R\< T \> , R\< T \>^+)$
    represents $\set{\abs{T} \leq 1}$ over $\AFF / \SPA(R , R^+)$.
  \end{proof1}
\end{lem}

\begin{eg}[The complement of $\bG_m$ in $\bA^1$]
  
  We continue to work over $\SPA(\bQ_p , \bZ_p)$ 
  for simplicity. 
  Let $\bG_m := \set{\abs{t} \neq 0} \subs \bA^1$.
  Although $\bA^1 = \widetilde{\SPA}(\bQ_p[T] , \bZ_p)$ is not affinoid,
  we can still compute a complete Huber pair which represents this functor.
  \begin{enumerate}
    \item The rings and ideal of definition for $(\bQ_p[T] , \bZ_p)$
    are $p \bZ_p \subs \bZ_p \subs \bQ_p[T]$.
    \item The topological ring with its ideal and ring of definition are
    $p \bZ_p \subs \bZ_p \subs \bQ_p[T , 1/T]$.
    \item Since $\bZ_p$ is already $p$-adically complete,
    the completed Huber ring is still $\bQ_p[T , 1 / T]$.
    \item Integral closure of $\bZ_p$ in $\bQ_p[T , 1 / T]$ is just $\bZ_p$.
    This is closed so the new completed Huber pair is
    $(\bQ_p[T , 1 / T] , \bZ_p)$.
  \end{enumerate}
  $\bG_m \simeq \widetilde{\SPA}(\bQ_p[T , 1/T] , \bZ_p)$
  is a filtered colimit of closed annuli centred at the origin
  along rational localizations and thus is an adic space.

  Question : what is the complement of $\bG_m$?
  In scheme theory, this is $\SPF \bQ_p[[T]]$.
  To understand what's going on,
  we first try the complement of $\bG_m \cap \bD$.
  Let $f : \SPA(A , A^+) \to \bD$ be a morphism of affinoids.
  This lies in the complement when for all $n \geq 0$,
  the preimage of $\set{\abs{p^n / T} \leq 1}$ in $\SPA(A , A^+)$ is empty.
  We have \begin{cd}
    {\mathrm{Spa}(A , A^+)} & {\mathrm{Spa}(\mathbb{Q}_p\langle T \rangle , \mathbb{Z}_p\langle T \rangle)} \\
    {\set{\vert p^n / f \vert \leq 1}} & {\set{\vert p^n / T \vert \leq 1}}
    \arrow["f", from=1-1, to=1-2]
    \arrow[from=2-1, to=1-1]
    \arrow["\lrcorner"{anchor=center, pos=0.125, rotate=90}, draw=none, from=2-1, to=1-2]
    \arrow[from=2-1, to=2-2]
    \arrow[from=2-2, to=1-2]
  \end{cd}
  Since $A$ is sheafy, $\set{\abs{p^n / f} \leq 1}$ is an affinoid.
  \textbf{TODO the topological ring $A\< X \> / (X g - 1)$
  is Huber for any $g \in A$ 
  and we have an isomorphic of complete Huber rings}
  \[
    A\< 1 / g \> \simeq A\< X \> / (X g - 1)
  \]
  Claim : this is the zero ring iff $g \in A^{\circ\circ}$.
  Assuming it is zero,
  we have $h \in A\< X \>$ with $(X g - 1) h = 0$.
  Writing $h = \sum_{n = 0}^\infty h_n T^n$ and expanding,
  we find $h_n = g^n$.
  Since $\lim_{n \to \infty} h_n = 0$ we have $g \in A^{\circ\circ}$.
  Conversely if $g \in A^{\circ\circ}$ then
  any $h \in A\<X\> / (Xg - 1)$ can be written as
  $h = g^{-n} h g^n$.
  But since $g^{-1}$ is power bounded in $A\< 1 / g \>$,
  $g^{-\bN} h$ is bounded and so 
  $h \in \bigcap_{n \geq 0} g^{-\bN} h g^n = (0)$.

  We use the above for $g = p^{-n} f$
  and obtain that $f$ lands in the complement of 
  $\set{\abs{p^n / T} \leq 1}$ iff 
  $p^{-n} f \in A^{\circ\circ}$.
  So $f$ lands in the complement of $\bG_m \cap \bD$
  iff $f \in \bigcap_{n \geq 0} p^{n} A^{\circ\circ} 
  = \bigcap_{n \geq 0} p^{n} A^{\circ\circ} A^{\circ}$.
  The same result happens for larger closed disks
  so the complement of $\bG_m$ is 
  \[
    \SPA(A , A^+) \mapsto \bigcap_{n \geq 0} p^n A^\circ
  \]
  This is the $\dagger$-nilradical!
\end{eg}

\section{The Fargues--Fontaine curve is an adic space}

We can now be precise about $Y_S$ being an adic space.
Let $\AFF\PERF \subs \AFF_{\TATE}$ denote 
the category of affinoid perfectoid characteristic $p$.
\begin{prop}
  
  Let $S = \SPA (R , R^+) \in \AFF\PERF$.
  Choose $\pi \in R^{\times} \cap R^{\circ\circ}$
  so that $R = R^+[1 / \pi]$.
  Define \[
    Y_S := \set{p [\pi] \neq 0} \subs \widetilde{\SPA}(W(R^+) , W(R^+))
  \]
  which we restrict to a presheaf on $\AFF_\TATE$.
  Then $Y_S$ is the filtered colimit of
  Tate affinoids along rational localizations 
  and hence an analytic adic space.
  This is independent of the choice of $\pi$.
  Furthermore,
  if we base change to $\bQ_p^\infty := \bQ_p(p^{1 / p^\infty})$,
  then \[
    Y_S \times_{\SPA(\bQ_p , \bZ_p)} \SPA(\bQ_p^\infty , \bZ_p^\infty)
  \]
  is a perfectoid space whose tilt is isomorphic to 
  the perfectoid punctured unit disk over $S$
  \[
    S \times_{\SPA \bF_p} 
    \SPA (\bF_p((t^{1 / p^\infty})) , \bF_p[[t^{1 / p^\infty}]]^\wedge_t)
  \]
\end{prop}
We try to give extra detail in constructing the atlas 
so that the reader knows 
the precise amount of data that goes into constructing an adic space.
\begin{proof}
  For $\SPA(A , A^+) \in \AFF_\TATE$,
  a morphism $\SPA(A , A^+) \to \set{p [\pi] \neq 0}$
  is equivalent to the data of 
  \begin{enumerate}
    \item a continuous ring morphism $\al : W(R^+) \to A$ such that
    \item $\al W(R^+) \subs A^+$
    \item $\al([\pi]) \in A^\times$
    \item $\al(p) \in A^\times$
  \end{enumerate}
  But $p$ is topologically nilpotent in $W(R^+)$ 
  so for all $b > 0$ there exists
  $a > 0$ with $\al(p)^a / \al([\pi])^b \in A^+$.
  Similarly, for all $d > 0$ there exists $c > 0$ with
  $\al([\pi])^c / p^d \in A^+$.
  Thus we have \begin{align*}
    Y_S &:= \set{p [\pi] \neq 0} = 
    \bigcup_{a, b, c, d > 0} Y_{S , [p^{- a / b} , p^{- d / c}]} \\
    Y_{S , [p^{- a / b} , p^{- d / c}]} 
      &:= \set{\abs{p^a} \leq \abs{[\pi]^b} \neq 0} \cap 
    \set{\abs{[\pi]^c} \leq \abs{p^d} \neq 0}
  \end{align*}
  The union is a presheaf filtered colimit.
  We show that $Y_{S , [p^{- a / b} , p^{- d / c}]}$
  is an affinoid over $\SPA(\bQ_p , \bZ_p)$
  and for inclusion of such intervals
  \begin{align*}
    I \subs J
    \,\,\,\,\implies \,\,\,\,
    Y_{S, I} \subs 
    Y_{S , J}
    \text{ rational localization}
  \end{align*}
  (Rational localization) 
  Follows from the description of $Y_{S , I}$.

  (Affinoid) We have \begin{align*}
    Y_{S, [p^{- a / b} , p^{- d / c}]}
    = \set{
      \abs{{p}^{a + d}} , \abs{[\pi]^{c}[\pi]^b} \leq \abs{[\pi]^b p^d \neq 0}
      }
  \end{align*}
  Following the procedure to compute rational localization, we get : 
  \begin{enumerate}
    \item The topology on $W(R^+)$ is the so-called weak topology,
    which comes from the bijection \[
      (R^+)^{\bN} \map{\sim}{} W(R^+) , (x_n) \mapsto \sum_{n = 0}^\infty [x_n] p^n
    \]
    taking the product topology on the LHS with each $R^+$ 
    the $\pi$-adic topology.
    This turns out to be precisely the $(p , [\pi])$-adic topology
    and $\pi$-adic completeness of $R^+$ implies
    $(p , [\pi])$-adic completeness of $W(R^+)$.
    A proof in the case of $R = F , R^+ = F^\circ$
    with $F$ a characteristic $p$ perfect complete NA field,
    see \cite[Prop. 1.4.11]{FF18}.

    \item The underlying ring is 
    \[
      B_S^b := W(R^+)[1 / [\pi]^b p^d] = W(R^+)[1 / [\pi]^b , 1 / p^d]
    \]

    \item The ring of definition is
    $W(R^+)\sqbrkt{\frac{p^{a + d}}{[\pi]^b p^d} , 
      \frac{[\pi]^{c}[\pi^{b}]}{[\pi]^b p^d}}
    = W(R^+)\sqbrkt{\frac{p^a}{[\pi]^b} , \frac{[\pi]^c}{p^d}}$.

    \item The ideal of definition is
    $(p , [\pi]) W(R^+)\sqbrkt{\frac{p^a}{[\pi]^b} , \frac{[\pi]^c}{p^d}}$.
    But since we have 
    $(p^a) \subs (p , [\pi])$
    and $(p , [\pi])^{2 c} \subs (p)$,
    an equivalent ideal of definition is 
    $pW(R^+)\sqbrkt{\frac{p^a}{[\pi]^b} , \frac{[\pi]^c}{p^d}}$.

    \item Our completed ring of definition is
    thus the $p$-adic completion \[
      B_{S , \sqbrkt{p^{- a / b} , p^{- d / c}}, 0} := 
      W(R^+)\sqbrkt{\frac{p^a}{[\pi]^b} , \frac{[\pi]^c}{p^d}}^\wedge_p
    \]
    equipped with the $p$-adic topology.

    \item Our completed Huber ring is \[
      B_{S , \sqbrkt{p^{- a / b} , p^{- d / c}}} := 
        W(R^+)\sqbrkt{\frac{p^a}{[\pi]^b} , \frac{[\pi]^c}{p^d}}^\wedge_p
        \sqbrkt{\frac{1}{[\pi]^b} , \frac{1}{p^d}}
      = W(R^+)\sqbrkt{\frac{p^a}{[\pi]^b} , \frac{[\pi]^c}{p^d}}^\wedge_p
      \sqbrkt{\frac{1}{p}}
    \]
    because $1 = \frac{1}{p^a} \frac{p^a}{[\pi]^b} [\pi]^b$
    implies inverting $p$ already makes $[\pi]^b$ invertible.

    \item Since the integral elements of $W(R^+)$ was 
    just $W(R^+)$,
    to compute the new set of integral elements,
    one must first take integral closure of 
    $W(R^+)\sqbrkt{\frac{p^a}{[\pi]^b} , \frac{[\pi]^c}{p^d}}$
    inside $B^b_S$,
    then take topological closure
    inside $B_{S , \sqbrkt{p^{- a / b} , p^{- d / c}}, 0}$.
    \textbf{This is probably just the same ring again
    but I struggle to compute integral closures.}
    Since $p$ in topologically nilpotent in $W(R^+)$
    and $p \in B_{S , [p^{- N} , p^{- 1 / M}]}^\times$,
    our completed Huber pair receives a unique morphism
    from $(\bQ_p , \bZ_p)$.
  \end{enumerate}
  For sheafiness,
  it suffices to show the above complete Tate--Huber pair is sousperfectoid.
  Consider the extension of NA fields $\bQ_p \to \bQ_p(p^{1 / p^\infty})$.
  Since $\bQ_p$ is discretely valued,
  any closed subspace of a $\bQ_p$-Banach space has a complement.
  \cite[Prop. 10.5]{Sch13}
  This gives a splitting of $\bQ_p \to \bQ_p(p^{1 / p^\infty})$
  in topological $\bQ_p$-vector spaces,
  which gives a splitting of the base change \[
    B_{S , \sqbrkt{p^{- a / b} , p^{- d / c}}} \to 
    \bQ_p(p^{1 / p^\infty}) \widehat{\otimes}_{\bQ_p} 
    B_{S , \sqbrkt{p^{- a / b} , p^{- d / c}}}
    =: B
  \]
  in topological $B_{S , \sqbrkt{p^{- a / b} , p^{- d / c}}}$-modules.
  We are now reduced to the final statement.
  
  (Base change to $\bQ_p^\infty$ is perfectoid)
  What's useful is that 
  the topology on $B_{S , I}$ actually comes from a power-multiplicative norm.
  \begin{lem}
    
    There exists a power-multiplicative norm $\abs{\_}_R$ on $R$
    which induces its topology.
    Choose such a norm.
    For $0 < \rho < 1$,
    define \[
    \abs{\_}_\rho : B^b_{S} \to \bR_{\geq 0} , 
    x = \sum_{n >> -\infty} [x_n] p^n 
    \mapsto \sup_{n \in \bZ} \abs{x_n}_R \rho^n
    \]
    Then this is a power-multiplicative norm
    making $B^b_S$ into a normed $\bQ_p$-algebra.
    For $I = [\rho_1 , \rho_2] \subs (0,1)$,
    completing w.r.t. $\abs{\_}_{\rho_1} , \abs{\_}_{\rho_2}$
    is the same as completing w.r.t.
    $\abs{\_}_I := \max(\abs{\_}_{\rho_1} , \abs{\_}_{\rho_2})$.
    \cite[Section 7.1.1]{Far24}
  \end{lem}
  Furthermore, it should be the case
  when $\rho_1 = \abs{[\pi]}^{a / b} , \rho_2 = \abs{[\pi]}^{d / c}$
  then this completion matches what we previously computed\footnote{
    Neither \cite[Section 5.1]{KL15} nor \cite[Prop. II.1.1]{FS24}
    is clear so this is my guess.
  }
  \begin{align*}
    \set{x \in B^b_F \st \abs{x}_I \leq 1} = 
    W(R^+)[p^a / [\pi]^b , [\pi]^c / p^d]\\
  \end{align*}
  Since $\bQ_p^\infty$ and $B_{S , I}$ are both 
  $\bQ_p$-Banach algebras with power-multiplicative norms,
  we can compute their completed tensor product easily : 
  take closed unit balls which are $\bZ_p$-algebras,
  take $p$-adically completion, then invert $p$.
  I'm hoping the completed tensor product (the projective one) of
  two power-multiplicatively normed Banach algebras
  give another power-multiplicatively norm.
  This gives uniformity of the base change to $\bQ_p^\infty$.

  It remains to show the surjectivity of Frobenius on $B^\circ / (p)$.
  Whatever $B^\circ$ is,
  its reduction mod $p$ will be \[
    B^\circ / (p) \simeq 
    \bZ_p[p^{1/p^\infty}]^\wedge_p / (p) {\otimes}_{\bF_p}
    W(R^+)[p^a / [\pi]^b , [\pi]^c / p^d]^\wedge_p / (p)
  \]
  We get \[
    B^\circ / (p) \simeq \bF_p[t^{1 / p^\infty}] / (t) \otimes_{\bF_p}
      R^+ / (\pi^c)
      \simeq (R^+)[t^{1 / p^\infty}] / (\pi^c , t) 
  \]
  Since $R$ is perfect and $R^+$ is integrally closed,
  $R^+$ is perfect as well.
  This implies $R^+ / (\pi^c)$ has surjective Frobenius
  and hence $B^\circ / (p)$ has surjective Frobenius.

  (The tilt of the base change)
  % Claim : The Frobenius $R^+ / (\pi^c) \to R^+ / (\pi^{c p})$ 
  % is an isomorphism of rings.
  % $R^+$ being perfect implies surjective.
  % Now for $x \in R^+$ with $x^p = \pi^{c p} y$ with $y \in R^+$
  % we have $(x / \pi^c)^p = y \in R^+$ and hence $x / \pi^c \in R^+$
  % so injectivity is true.
  % Completeness of $(R , R^+)$ implies $R^+$ is $\pi^c$-adically complete.
  % Thus
  % $(\pi^c , t)^{2p} \subs (\pi^{c p} , t^p) \subs (\pi^c , t)$
  % implies we have isomorphism of rings
  % \begin{align*}
  %   B^{\flat , \circ} := 
  %   \LIM_{x \mapsto x^p} B^\circ
  %   \simeq \LIM_{x \mapsto x^p} B^\circ / (p) 
  %   \simeq R^+[t^{1 / p^\infty}]^{\wedge}_{(\pi^c , t)}
  %   \simeq R^+ \widehat{\otimes}_{\bF_p} \bF_p[[t^{1 / p^\infty}]]^\wedge_t
  % \end{align*}
  We write $Y_{S} \widehat{\otimes}_{\bQ_p} \bQ_p^\infty$
  for the base change.
  Any morphism from an affinoid into 
  $Y_{S} \widehat{\otimes}_{\bQ_p} \bQ_p^\infty$
  must factor through some 
  $Y_{S , I} \widehat{\otimes}_{\bQ_p} \bQ_p^\infty$.
  Since the tilt of a rational localization
  is the ``same'' rational localization of the tilt \cite[Theorem 7.1.1]{SW20},
  it follows that it suffices to show \[
    (Y_{S , I} \widehat{\otimes}_{\bQ_p} \bQ_p^\infty)^\flat
    \simeq \bD_{S , I}
  \]
  where $\bD_{S , I}$ is the
  adic closed annuli over $S$ with inner and outter radii specified by $I$.

  Notice that since $\bQ_p$ is Tate,
  the morphisms $\bQ_p \to \bQ_p^\infty$ and 
  $\bQ_p \to B_{S , I}$ of complete Tate rings are adic.
  It follows from \cite[Prop. 5.1.5.(2), Rmk. 5.1.6]{SW20}
  that $\bQ_p^\infty \widehat{\otimes}_{\bQ_p} B_{S , I}$
  is the pushout in the category of complete Tate rings.
  Thus for any $T = \SPA(A , A^+) \in \AFF\PERF$, 
  \begin{enumerate}
    \item a morphism of affinoids 
    $T \to (Y_{S , I} \widehat{\otimes}_{\bQ_p} \bQ_p^\infty)^\flat $
    is equivalent to
    \item an untilt $T^\sharp$ together with a morphism of affinoids
    $T^\sharp \to Y_{S , I} \widehat{\otimes}_{\bQ_p} \bQ_p^\infty$,
    which is equivalent to
    \item an untilt $T^\sharp$ together with a pair of morphisms
    of affinoids 
    \begin{align*}
      T^\sharp \to \SPA (\bQ_p^\infty , \bQ_p^{\infty , \circ}) \\
      T^\sharp \to \SPA (B_{S , I} , B_{S , I}^+)
    \end{align*}
    which is equivalent to
    \item by the tilting correspondence,
    a morphism of affinoids 
    \[
      T \to \SPA (\bF_p((t^{1 / p^\infty})) , \bF_p[[t^{1 / p^\infty}]]^\wedge_t)
    \]
    and an untilt $T^\sharp$ together with a continuous ring morphism 
    \[
      W(R^+)\sqbrkt{\frac{p^a}{[\pi]^b} , \frac{[\pi]^c}{p^d}}^\wedge_p
      \to \cO(T^\sharp)^+
    \]
    Such a continuous ring morphism
    is equivalent to a continuous ring morphism 
    $W(R^+) \to \cO(T^\sharp)^+$ such that
    $p^a / [\pi]^b , [\pi]^c / p^d \in \cO(T^\sharp)^+$ as well.
    So we have a further equivalence to 
    \item a morphism of affinoids 
    \[
      T \to \SPA (\bF_p((t^{1 / p^\infty})) , \bF_p[[t^{1 / p^\infty}]]^\wedge_t)
    \]
    and a continuous ring morphism
    \[
      \al : R^+ \to A^+
    \]
    such that $t^a / \al(\pi^b) , \al(\pi^c) / t^d$ into $A^+$,
    where we have used $t = p^\flat$
    and the following facts : 
    \begin{itemize}
      \item (Fact 1) For $R^+$ perfect and $A^\sharp$ perfectoid,
      we have \begin{align*}
        \set{\text{cts $W(R^+) \to A^{\sharp , \circ}$}}
        &\map{\text{mod $p$}}{\sim}
        \set{\text{cts $R^+ \to A^{\sharp , \circ} / (p)$}} 
        & \text{theory of Witt vectors} \\
        &\map{}{\sim} \set{\text{cts $R^+ \to A^\circ$}}
        &\text{$R^+$ perfect, $A$ perfectoid}
      \end{align*}
      \item (Fact 2) \cite[Lem. 6.2.5]{SW20} Since $A^\sharp$ is perfectoid, 
      we have an isomorphism of multiplicative monoids 
      \begin{cd}
        {\varprojlim_{x \mapsto x^p} A^{\sharp , +}} & {\varprojlim_{x \mapsto x^p} A^{\sharp , \circ}} & {\varprojlim_{x \mapsto x^p} A^\sharp} \\
        {A^+} & {A^\circ} & A
        \arrow["\subseteq"{description}, draw=none, from=1-1, to=1-2]
        \arrow["\sim", from=1-1, to=2-1]
        \arrow["\subseteq"{description}, draw=none, from=1-2, to=1-3]
        \arrow["\sim", from=1-2, to=2-2]
        \arrow["\sim", from=1-3, to=2-3]
        \arrow["\subseteq"{description}, draw=none, from=2-1, to=2-2]
        \arrow["\subseteq"{description}, draw=none, from=2-2, to=2-3]
      \end{cd}
      The middle vertical map has an explicit inverse 
      $x \mapsto (x^{\sharp} , (x^{1 / p})^\sharp , \cdots)$.
    \end{itemize}
    Finally, this is equivalent to 
    \item a morphism of affinoids 
    $T \to \bD_{S , I} := \{\abs{t^a} \leq \abs{\pi^b} \neq 0 \} 
      \cap \{\abs{\pi^c} \leq \abs{t}^d \neq 0\}$
  \end{enumerate}
\end{proof}
% \begin{rmk}
%   The case of $S = \SPA (F , F^\circ)$ 
%   a perfectoid field was first considered in \cite{FF18}.
%   The notation in our proof is compatible with the notation in \cite{FF18}.
%   There is a norm-theoretic definition in this case.
%   \cite[Def. 1.6.2]{FF18} defines $B_{F , I}$ for closed interval 
%   $I \subs (0 , 1)$ using completion w.r.t. a family of norms.
%   The construction we gave following \cite[Theorem 3.1.3]{SW20} agrees
%   with the computation in \cite[Example 1.6.3]{FF18}.
% \end{rmk}

% By being more careful with the above proof,
% one can define $Y_{S , I}$ for compact intervals $I \subs (0,1)$
% with rational end points and show that they are affinoid.

\begin{prop}
  
  Let $S = \SPA(A , A^+) \in \AFF\PERF$.
  By the theory of Witt vectors,
  the Frobenius automorphism on $S$ induces
  an automorphism $\varphi$ of $\widetilde{\SPA}(W(R^+) , W(R^+))$,
  and hence on $Y_S$.
  Define \[
    X_S := Y_S / \varphi^\bZ
  \]
  where the quotient is as sheaves over $\AFF$ for the analytic topology.
  Then $X_S$ is an adic space.
  In fact, 
  one can obtain $X_S$ as the quotient of
  $Y_{S , [p^{- p} , p^{-1}]}$ by identifying the rational subspaces
  $\varphi : Y_{S , [p^{-p} , p^{-p}]} \simeq Y_{S , [p^{-1} , p^{-1}]}$.
  In particular, $X_S$ is quasi-compact.
\end{prop}
The picture one should have is of the Tate elliptic curve:
closed annuli on the punctured open unit disk
are identified by radial scaling, forming a torus
which looks like the complex points of an elliptic curve.
\begin{proof}
  Concretely, $\varphi : W(R^+) \to W(R^+)$ maps 
  \[
    x = \sum_{n = 0}^\infty [x_n] p^n \mapsto 
    \sum_{n = 0}^\infty [x_n^p] p^n
  \]
  So if we start with $\SPA(A , A^+) \to Y_{S , [p^{- N} , p^{- 1 / M}]}$,
  which means on $\SPA(A , A^+)$ we have
  \[
    \abs{p^a} \leq \abs{[\pi]^b} \text{ and } \abs{[\pi]^c} \leq \abs{p^d}
  \]
  Composing with $Y_S \to Y_S$ coming from $\varphi$ on algebras,
  we get the following condition on $\SPA(A , A^+)$
  \[
    \abs{p^a} \leq \abs{[\pi]^{bp}} \text{ and } \abs{[\pi]^{cp}} \leq \abs{p^d}
  \]
  which is equivalent to 
  $\SPA(A , A^+) \to Y_{S , [p^{- a / b p} , p^{- d / c p}]}$
  It follows that $\varphi$ induces an isomorphism \[
    Y_{S , [p^{- a / b} , p^{- d / c}]}\map{\varphi}{\sim} 
    Y_{S , [p^{- a / bp} , p^{- d / cp}]} 
  \]
  i.e. ``the closed annuli moves towards radius 1''.
  For closed intervals $I \subs (0,1)$ of the form
  $I = [p^{- a / b} , p^{- d / c}]$
  we have commutative diagrams \begin{cd}
    {Y_{S , I}} & {Y_{S , \varphi I}} \\
    {Y_{S , J}} & {Y_{S , \varphi J}}
    \arrow["\varphi", from=1-1, to=1-2]
    \arrow["\sim"', from=1-1, to=1-2]
    \arrow["{\text{rat. loc.}}"', from=1-1, to=2-1]
    \arrow["{\text{rat. loc.}}", from=1-2, to=2-2]
    \arrow["\varphi"', from=2-1, to=2-2]
    \arrow["\sim", draw=none, from=2-1, to=2-2]
  \end{cd}
  Identifying these affinoids under the action of $\varphi$
  thus gives another filtered system of affinoids
  along rational localizations.
  It follows that the presheaf colimit is the sheaf colimit
  and is an adic space.
  The universal property is clear.

  (Quasi-compact) 
  For compact intervals $I \subs (0,1)$
  such that $I \cap I^{1 / p} = \varnothing$,
  the morphism $Y_{S , I} \to X_S$ is an open immersion.
  Since $(0,1) = \bigcup_{n \in \bZ} [p^{- p / p^{n}} , p^{-1 / p^{n}}]$,
  the result follows.
\end{proof}

\begin{prop}
  Let $S = \SPA(R , R^+) \in \AFF\PERF$. Then we have
  \[
   Y_S^\diamond = S \times \SPD \bQ_p
 \]\[
   X_S^\diamond = (S \times \SPD \bQ_p) / (\varphi^\bZ \times \id)
 \]
 \cite[Prop. II.1.17]{FS24}
 This is the rational version of \cite[Prop. 11.2.1]{SW20}.
\end{prop}
\begin{proof}
  The description of $X_S^\diamond$ follows from
  that of $Y_S^\diamond$.
  For $Y_S^\diamond$, note that we have already proved
  \begin{cd}
    S & {S \times \SPD \bQ_p} 
      & {(Y_S \widehat{\otimes}_{\mathbb{Q}_p} \mathbb{Q}_p^\infty)^\diamond} \\
    \bullet & {\mathrm{Spd}\,\mathbb{Q}_p} & {\mathrm{Spd}\,\mathbb{Q}_p^\infty}
    \arrow[from=1-1, to=2-1]
    \arrow[from=1-2, to=1-1]
    \arrow["\lrcorner"{anchor=center, pos=0.125, rotate=-90}, draw=none, from=1-2, to=2-1]
    \arrow[from=1-2, to=2-2]
    \arrow[from=1-3, to=1-2]
    \arrow["\lrcorner"{anchor=center, pos=0.125, rotate=-90}, draw=none, from=1-3, to=2-2]
    \arrow[from=1-3, to=2-3]
    \arrow[from=2-2, to=2-1]
    \arrow[from=2-3, to=2-2]
  \end{cd}
  To compute $S \times \SPD \bQ_p$,
  by descent in toposes 
  it suffices to compute $S \times \SPD \bQ_p^\infty$
  with its $\bZ_p^\times$ equivariant projection
  to $\SPD \bQ_p^\infty$.
  The computation 
  $(Y_S \widehat{\otimes}_{\mathbb{Q}_p} \mathbb{Q}_p^\infty)^\diamond \simeq 
  S \times D_{S}$ did exactly this.
%  Let $T = \SPA(A , A^+) \in \AFF\PERF$.
%  A morphism $T \to Y_S^\diamond$ is
%  the data of : 
%  \begin{enumerate}
%    \item[(i)] an untilt $T^\sharp = \SPA(A^\sharp , A^{\sharp , +})$
%    \item[(ii)] a continuous ring morphism $\al : W(R^+) \to A^\sharp$ such that
%    \item[(iii)] $\al W(R^+) \subs A^{\sharp , +}$
%    \item[(iv)] $[\pi]$ lands in $A^{\sharp , \times}$
%    \item[(v)] $p$ lands in $A^{\sharp , \times}$
%  \end{enumerate}
%  On the other hand a morphism $T \to S \times \SPD \bQ_p$
%  consists of \begin{enumerate}
%    \item[(a)] an untilt $T^\sharp = \SPA(A^\sharp , A^{\sharp , +})$
%    \item[(b)] $p \in A^{\sharp , \times}$
%    \item[(c)] $A^{\sharp , +}$ contains $p$
%    \item[(d)] a continuous ring morphism $R \to A$
%    sending $R^+$ to $A^+$.
%  \end{enumerate}
%  The data of (i) and (a) are the same.
%  Using facts (1) and (2) from before,
%  (ii), (iii), (iv) together is equivalent to (c) and (d) together.
%  Finally, under (i), (ii), (iii), (iv),
%  we have (v) iff (b).
\end{proof}

\textbf{Warning} : the morphism $Y_S^\diamond \to S$ does \emph{not}
come from a morphism $Y_S \to S$ of adic spaces.
(Indeed $p$ is invertible on $Y_S$ whilst $p$ is zero on $S$.)
Furthermore, $X_S^\diamond$ does not have a morphism to $S$
even as pro-étale sheaves over $\AFF\PERF$.
Nonetheless, a morphism $T \to S$ in $\AFF\PERF$ induces 
morphisms of adic spaces $Y_T \to Y_S$ such that
if $S \to Y_S^\diamond$ corresponds to an untilt $S^\sharp \to Y_S$
then by \cite[Theorem 6.2.11]{SW20}
the untilt $T^\sharp \to S^\sharp$ under the tilting equivalence
fits in a cartesian square : 
\begin{cd}
  {S^\sharp} & {T^\sharp} \\
	{Y_S} & {Y_T}
	\arrow[from=1-1, to=2-1]
	\arrow[from=1-2, to=1-1]
	\arrow["\lrcorner"{anchor=center, pos=0.125, rotate=-90}, draw=none, from=1-2, to=2-1]
	\arrow[from=1-2, to=2-2]
	\arrow[from=2-2, to=2-1]
\end{cd}

We delay addressing the elephant in the room :  
\emph{In what sense is $X_S$ a relative curve over $S$?}

\section{Vector bundles on the Fargues--Fontaine curve}

We study vector bundles on $X_S$.
In the motivation,
we got the idea for $X_S$ from the space
$\SPEC W(\overline{\bF_p})[1 / p] / \varphi^\bZ$.
Vector bundles on the latter are called 
\emph{isocrystals over $\overline{\bF_p}$}.
These are very well understood.
We write $\breve{\bQ}_p$ for $\breve{\bQ}_p$.

\begin{prop}[Dieudonné classification of isocrystals]
  
  Define the category $\varphi\MOD(\breve{\bQ}_p)$ 
  of isocrystals over $\overline{\bF_p}$ as 
  the category of finite dimensional $\breve{\bQ}_p$-vector spaces 
  $V$ together with a linear isomorphism \[
    \varphi_V : \varphi^* V \simeq V
  \]
  Then $\varphi\MOD(\breve{\bQ}_p)$ is semisimple with
  simple objects, up to isomorphism, given by \[
    D(\la) := (\breve{\bQ_p}[T] / (T^r - p^d) , T)
  \]
  with $\la = d / r \in \bQ$ , $r > 0$ , $(d , r) = 1$.
  The division algebra $\END D(\la)$ over $\bQ_p$
  is central with invariant $- [\la] \in \bQ / \bZ \simeq \mathrm{Br}(\bQ_p) $.
\end{prop}
% \begin{proof}
%   First, one classifies the semistable objects.
%   Using twists by rank one isocrystals,
%   one reduces to the case of showing any semistable slope 0 $V$ 
%   is isomorphic to direct sum of $D(0)$.
%   This follows by showing
%   \begin{itemize}
%     \item $V$ has a subobject isomorphic to $D(0)$.
%     \item all extensions of $D(0)$ by itself are trivial.
%   \end{itemize}

%   Using the classification of semistable objects of $\varphi\MOD_{\breve{E}}$,
%   one notes that the HN formalism applies with $\deg$ swapped to $-\deg$.
%   (i.e. Given semistable $V$ with slope $\la$, all strict subobjects
%   also have slope $\la$.)
%   Apparently this implies the HN filtration is split,
%   but I don't understand why.
% \end{proof}
A proof of the above uses the so-called HN formalism,
which appears in the study of vector bundles
on curves in algebraic geometry.
\begin{prop}[Harder--Narasimhan formalism]
  
  Let $C$ be an exact category\footnote{
    Here we mean in the sense of Quillen.
    This can be defined as full additive subcategories
    of abelian categories which are
    closed under isomorphisms and extensions.
  } with two functions 
  \[
    \deg : \mathrm{Obj}\,C \to \bZ  
  \]
  \[
    \mathrm{rk} : \mathrm{Obj}\,C \to \bN  
  \]
  which are additive on short exact sequences.
  Assume the existence of an exact faithful functor
  \[
    F : C \to A  
  \]
  to an abelian category $A$ and an additive function 
  $\mathrm{rk} : \mathrm{Obj}\,A \to \bN$ extending 
  $\mathrm{rk} : \mathrm{Obj}\,C \to \bN$ along $F$.
  Furthermore, require three conditions : 
  \begin{enumerate}
    \item $F$ induces for each $\cE \in C$ a bijection 
    \[
      \set{\text{strict subobjects of }\cE} \simeq 
      \set{\text{subobjects of }F(\cE)}  
    \]
    where strict subobjects are subobjects fitting in a SES.
    \item For $\cE \in C$\, $\mathrm{rk}\,\cE = 0 $ iff $\cE = 0$.
    \item Given $u : \cE \to \cF$ in $C$ with $F(u)$ an isomorphism,
    then $\deg \cE \leq \deg \cF$ with equality iff $u$ is an isomorphism.
  \end{enumerate}
  Given the above situation,
  define the \emph{slope} of $\cE \in C$ to be 
  \[
    \mu(\cE) := \deg(\cE) / \mathrm{rk}\,\cE \in \bQ \cup \set{\infty}
  \]
  An object $\cE \in C$ is called \emph{semistable} when
  for all non-zero strict subobjects $\cF \subs \cE$
  we have $\mu(\cF) \leq \mu(\cE)$.
  Then for each $\cE \in C$ there exists a filtration
  \[
    0 = \cE_0 \subs \cdots \subs \cE_r = \cE  
  \]
  unique with respect to the properties : 
  \begin{enumerate}
    \item $\cE_{i} / \cE_{i - 1}$ is semistable
    \item $\mu(\cE_{i} / \cE_{i - 1})$ strictly decreasing
    as $i$ increases.
  \end{enumerate}
  \cite[Section 5.5]{FF18}
\end{prop}
\begin{proof}
  Skipping over many details, existence is by induction on rank of $\cE$.
  Uniqueness is by proving that given $\cE, \cF$ semistable
  with $\mu(\cE) > \mu(\cF)$ then $C(\cE , \cF) = 0$.
  This can also be used to show the filtration is functorial in $\cE$.
\end{proof}
The point is that this applies to 
\begin{itemize}
  \item $C = A = \varphi\MOD_{\breve{\bQ}_p}$
  \item $\mathrm{rk}\,V:= \dim_{\breve{Q}_p} V$
  \item Using the observation \[
    \pi_0 \set{\text{rank 1 isocrystals}} \simeq
    {\breve{E}}^\times / \varphi\text{-conjugacy} \simeq \bZ
  \]
  define $\deg V := \deg \bigwedge^{\mathrm{rk}\,V} V$
\end{itemize}
To get the classification,
one must calculate what these semistable objects are
and show that the HN filtration splits.
This suggests some ideas : 
\begin{enumerate}
  \item [Q1] Is there a functor from
  isocrystals over $\overline{\bF_p}$ to vector bundles on $X_S$.
  \item [Q2] Does the HN formalism apply to vector bundles on $X_S$?
  If so, how does the classification compare with isocrystals?
\end{enumerate}
The answer to (Q1) is affirmative.
\begin{dfn}[$\varphi$-modules to vector bundles on the curve]
  
  Let $S = \SPA(R , R^+) \in \AFF\PERF$.
  The extension $\bar{\bF_p} \to R$ gives
  $\breve{\bQ}_p = W(\bar{\bF_p}) \to W(R^+)$ by functoriality.
  This induces a morphism \[
    Y_{S} \to \SPA (\breve{\bQ}_p , \breve{\bQ}_p^\circ)
  \]
  so we can pullback finite dimensional vector spaces over $\breve{\bQ}_p$
  to get vector bundles on $Y_{S}$.
  Define \[
    \varphi\MOD(\breve{\bQ}_p) \to \mathrm{VB}(X_{S})
  \]
  \[
    (V , \varphi_V) \mapsto \cE(V , \phi_V)
  \]
  where the latter is 
  obtained by taking the trivial vector bundle over $Y_{S}$ with fiber $V$,
  then descending it along $Y_{S} \to X_{S}$ by letting
  $\varphi$ act via $\varphi_V$.
  We write $\cO(d / r) := \cE({D(- d / r)})$ where $D( - d / r)$ is simple.
\end{dfn}
To study vector bundles,
the following GAGA result says we can WLOG work in scheme theory.
\begin{prop}[GAGA]
  
  Let $(X , \cO_X)$ be a locally ringed spectral space.
  Suppose we have a line bundle $\cO_X(1)$ such that 
  \begin{enumerate}
    \item there exists $N \geq 0$ such that for all $n \geq N$ 
    and all vector bundles $\cE$ on $X$, $\cE(n)$ is globally generated.
    \item for all vector bundles $\cE$ on $X$
    and for all $i > 0$
    there exists $n$ such that $H^i(X , \cE(n)) = 0$.
  \end{enumerate}
  Let $P := \bigoplus_{n \geq 0} H^0(X , \cO_X(n))$.
  This is a graded ring so we can define 
  $X^{\mathrm{alg}} := \mathrm{Proj} P$.
  Then there exists a morphism of locally ringed spaces
  $(X , \cO_X) \to X^{\mathrm{alg}}$ such that 
  \begin{enumerate}
    \item pullback induces an equivalences an equivalence of
    categories of vector bundles
    \item for any vector bundle $\cE$ on $X$ with
    corresponding vector bundle $\cE^{\mathrm{alg}}$ on $X^{\mathrm{alg}}$,
    for all $i \geq 0$
    we have an induced isomorphism \[
      H^i(X^{\mathrm{alg}}, \cE^{\mathrm{alg}}) \map{\sim}{}
      H^i(X , \cE)
    \]
  \end{enumerate}
  In this case, the tautological line bundle on $X^{\mathrm{alg}}$
  pulls back to $\cO_X(1)$.
  \cite[Prop. II.2.7]{FS24}
\end{prop}
The relative Fargues--Fontaine curves satisfy this.
\begin{prop}[Schematic relative Fargues--Fontaine curve]
  
  Let $S$ continue to be in $\AFF\PERF$.
  Let $\cE$ be a vector bundle on $X_S$.
  Then there exists $N \geq 0$ such that for all $n \geq N$
  we have a surjection for some $m \geq 0$
  \[
    \cO_{X_S}^m \twoheadrightarrow \cE(n)
  \]
  and moreover $H^{i > 0}(X_S , \cE(n)) = 0$.
  We write $X_S^{\mathrm{alg}}$ for the scheme
  obtained using GAGA.
\end{prop}
\begin{proof}
  We can show that if $\cE$ is a vector bundle over $X_S$,
  then $H^{i}(X_S , \cE) = 0$ for $i \neq 0 , 1$.
  For $\cO$-modules on sheafy $(A , A^+)$ associated to 
  finite $A$-modules, there is no higher cohomology.
  \cite[Theorem 2.5.20]{KL15}
  Combining this with the fact that pushforward of
  $\cO$-modules along open immersions of ringed spaces is exact,
  it suffices to find an open cover of $X_S$ by two affinoids.
  For $p$ odd, take $Y_{S , [p^{-p} , p^{- (p-1) / 2 p}]}$ and 
  $Y_{S , [p^{- (p-1)/2p} , p^{-1}]}$.
  For $p = 2$, take $Y_{S , [1 / 4 , 3 / 8]}$ and $Y_{S , [3 / 8 , 1 / 2]}$.

  See \cite[Theorem II.2.6]{FS24} for the ``ampleness'' of $\cO_{X_S}(1)$.
\end{proof}
We can now address the question : 
in what sense is $X_S$ a relative curve over $S$?
Fargues and Fontaine define curves as follows :
\begin{dfn}
  
  A curve is a smooth, Noetherian, connected, separated scheme $X$
  of Krull dimension one together with
  a positive integer $\deg x$ for every closed point $x \in X$.
  \cite[Def. 5.1.1]{FF18}

  Let $X$ be a curve.
  Consider the exact sequence \[
    1 \to \cO(X)^\times \to \cM(X)^\times \map{\mathrm{div}}{} \DIV X \to \PIC X \to 0
  \]
  Define \[
    \deg : \DIV X \to \bZ , \sum_{x \in \abs{X}} m_x x \mapsto 
    \sum_{x \in \abs{X}} m_x \deg(x)
  \]
  We say $X$ is \emph{complete} when $\deg$ factors through $\PIC X$,
  i.e. for any non-zero rational function $f$ we have $\deg \mathrm{div} f = 0$.
\end{dfn}
\begin{eg}
  Let $k$ be a field and $X$ a smooth projective curve over $k$
  in the usual sense.
  Then assigning every closed point $x$ the value
  $\deg x := [\kappa(x) : k]$ defines a curve in the sense of \cite{FF18}.
\end{eg}
We specialise to the case of a geometric point.
\begin{prop}
  
  Let $S = \SPA(F , F^\circ)$ where
  $F$ is a complete algebraically closed NA field of characteristic $p$.
  Then the following is true : 
  \begin{enumerate}
    \item The closed points of $X_F^{\mathrm{alg}}$
    biject with characteristic zero untilts of $F$ up to Frobenius.
    \item $X_F^\mathrm{alg}$ is a complete curve in the sense of \cite{FF18}
    when we set $\deg(x) = 1$ for all closed points.
    \item For any closed point $x_{F^\sharp}$ of $X_F^{\mathrm{alg}}$,
    the complement $X_{F}^{\mathrm{alg}}\setminus\set{x_{F^\sharp}}$
    is the spectrum of a PID.
  \end{enumerate}
  \cite[Theorem 6.5.2]{FF18}
\end{prop}
For general $S \in \AFF\PERF$,
one should think of $X_S$ as a family of curves
$X_{\ka(s)}$ where $s$ ranges over points of $S$
valued in completed algebraically closed NA fields (up to equivalence).

For $S$ a geometric point, we have the following answer to (Q2) and (Q3) :
\begin{prop}[Classification of vector bundles on the Fargues--Fontaine curve]
  
  Let $S = \SPA(F , F^\circ)$ where
  $F$ is a complete algebraically closed NA field of characteristic $p$.
  \begin{enumerate}
    \item \cite[Prop. II.2.12]{FS24} 
    The HN formalism applies to the category of vector bundles
    on $X_F$ with the usual notion of degree and rank.
    Thus every vector bundle $\cE$ has a unique exhaustive separating
    $\bQ$-indexed filtration with factors which are semi-stable
    with single slope and the slope strictly decreases 
    going up towards all of $\cE$.
    This is called the HN filtration and it is respected 
    by morphisms of vector bundles.
    \item \cite[Prop. II.2.13]{FS24} 
    For $F \to F^\prime$ a continuous extension of 
    complete algebraically closed NA fields of characteristic $p$,
    pullback along $X_{F^\prime} \to X_F$
    sends HN filtrations to HN filtrations.
    \item \cite[Prop. II.2.14]{FS24}
    The HN filtration splits and 
    if $\cE$ is semistable with single slope $\la$,
    then $\cE \simeq \cO_{X_F}(\la)^m$ for some $m \geq 0$.
    Thus any vector bundle $\cE$ on $X_F$ is
    a direct sum of vector bundles of the form 
    $\cO_{X_F}(\la)$ for $\la \in \bQ$.
  \end{enumerate}
\end{prop}
In particular,
the functor $(V , \varphi_V) \mapsto \cE(V , \varphi_V)$
induces a bijection on the sets of isomorphism classes of
isocrytals over $\overline{\bF_p}$ and vector bundles on $X_F$.

We will need the following computation of global sections : 
\begin{prop}
  Let $\la = d / r \in \bQ$ with $(d , r) = 1$ and $r > 0$.
  \begin{cd}
    & {d < 0} & {d = 0} & {0 < d} \\
    {H^0(X , \mathcal{O}(d / r) )} & 0 & {\mathbb{Q}_p} & {B^{\varphi^r = p^d} \neq 0} \\
    {H^1(X , \mathcal{O}(d / r) )} & {\text{big}} & 0 & 0
  \end{cd}
\end{prop}
\begin{proof}
  See \cite[Prop. 8.2.3]{FF18}.
  $H^0(X , \cO) = \bQ_p$ can be done by developing a theory of Newton polygons
  on the ring $B^b$.
\end{proof}

\section{Application : Étale fundamental group of the Fargues--Fontaine curve}

Fix $F$ a complete algebraically closed NA field of characteristic $p$.
We prove : 
\begin{prop}
  
  Base change along $X_F \to X_F^{\mathrm{alg}} \to \SPEC \bQ_p$
  induces an equivalence between finite étale morphisms
  over $X_F$ and finite étale algebras over $\bQ_p$.
  Consequently,
  we have \[
    \GAL(\overline{\bQ_p} / \bQ_p) \simeq \pi_1^{\text{ét}}(X_F , x)
  \]
\end{prop}
\begin{proof}
  We follow \cite[Theorem 13.5.7]{SW20}.
  By GAGA, we can just work with the schematic $X := X_F^{\mathrm{alg}}$.
  Let $\widetilde{X} \to X$ be a finite étale cover
  and let $\cE$ be the corresponding finite étale $\cO_{X}$-algebra.
  It suffices to show $\cE$ is a trivial vector bundle
  since $H^0(X , \cO) = \bQ_p$.

  Since $\widetilde{X} \to X$ is finite, flat, locally finitely presented,
  $\cE$ is locally free finite rank.
  By the classification of vector bundles on $X$,
  we can write $\cE \simeq \cO(\la_1) \oplus \cdots \oplus \cO(\la_s)$.
  It suffices to show all $\la_i = 0$.

  Since $\cE$ is a vector bundle,
  one has a well-defined trace map $\mathrm{tr} : \cE \to \cO_X$.
  Étaleness implies the pairing $\cE \otimes_{\cO_X} \cE \to \cE \to \cO_X$
  is non-generate 
  which exhibits $\cE$ as its own $\cO_X$-linear dual.
  \cite[\href{https://stacks.math.columbia.edu/tag/0BVH}{Tag 0BVH}]{stacks-project}
  Since $\det (\cE^\vee) \simeq \brkt{\det \cE}^{-1}$,
  we have $\deg \cE = \deg \cE^\vee = - \deg \cE$.
  This implies $\sum_{s} \la_s = 0$.
  
  Let $\la := \max(\la_s)$ so that $\la \geq 0$.
  It remains to show that $\la = 0$.
  Suppose for a contradiction that $\la > 0$.
  Then composition \[
    \cO(\la) \otimes \cO(\la) \to \cE \otimes \cE \map{\text{mul}}{} \cE
  \]
  gives a global section $f \in \HOM(\cO(\la)^{\otimes 2} , \cE)
  \simeq \HOM(\cO , \cE \otimes \cO(-2 \la)) = H^0(X , \cE \otimes \cO(- 2\la))$.
  \begin{lem}
  
    For $(V , \varphi_V) , (W , \varphi_W) \in \varphi\MOD(\breve{\bQ}_p)$,
    define \[
      (V , \varphi_V) \otimes (W , \varphi_W) := 
      (V \otimes W , \varphi_V \otimes \varphi_W)
    \]
    For $d_1 / r_1 , d_2 / r_2 \in \bQ$
    with $(d_i , r_i) = 1$ and $r_i > 0$,
    we have \[
      D\brkt{\frac{d_1}{r_1}} \otimes D\brkt{\frac{d_2}{r_2}}
      \simeq D\brkt{\frac{d_1}{r_1} + \frac{d_2}{r_2}}^{\oplus (r_1 , r_2)}
    \]
    Hence we have the following isomorphism of vector bundles on $X_S$ :
    \[
      \cO_{X_S}\brkt{\frac{d_1}{r_1}} \otimes \cO_{X_S}\brkt{\frac{d_2}{r_2}}
      \simeq 
      \cO_{X_S}\brkt{\frac{d_1}{r_1} + \frac{d_2}{r_2}}^{\oplus (r_1 , r_2)}
    \]
    \cite[Prop. 5.6.23]{FF18}
  \end{lem}
  By the lemma, 
  all slopes in the decomposition of $\cE \otimes \cO(- 2\la)$ are negative,
  so $H^0(X , \cE \otimes \cO(- 2\la)) = 0$.
  This implies that for all $f \in H^0(X , \cO(\la)) \subs H^0(X , \cE)$
  we have $f^2 = 0$.
  Since $\cE$ is étale, $H^0(X , \cE)$ is a reduced ring
  so $H^0(X , \cO(\la)) = 0$,
  which is a contradiction because $\la > 0$.
\end{proof}

% \section{Relation to $p$-adic Hodge theory}

% Let $C$ be a complete algebraically closed NA field of characteristic zero
% and set $F := C^\flat$.
% Choose a perfectoid pseudo-uniformizer for $C$
% giving rise to a pseudo-uniformizer $p^\flat \in F^\circ$
% such that we have a commutative square
% \begin{cd}
%   {W(F^\circ)} & {C^\circ} \\
% 	{F^\circ} & {F^\circ / (p^\flat) \simeq C^\circ / (p)}
% 	\arrow["\theta", from=1-1, to=1-2]
% 	\arrow[from=1-1, to=2-1]
% 	\arrow[from=1-2, to=2-2]
% 	\arrow[from=2-1, to=2-2]
% \end{cd}
% where as usual $\xi := p - [p^\flat]$ generates $\ker \theta$.
% Define \[
%   A_{\mathrm{cris}} := 
%     \brkt{W(F^\circ)\sqbrkt{\frac{\xi^n}{n!} \st n \geq 0}}^\wedge_p
% \]
% This is the $p$-adic completion of the divided power envelope of
% the map $W(F^\circ) \to C^\circ / (p)$.
% It is in fact the final pro-object in the cristalline site
% $\mathrm{Cris}((C^\circ / (p)) / \SPEC \bZ_p)$.
% We also have :
% \cite[Section 1.10.3]{FF18}
% \begin{prop}
%   \[
%   A_{\mathrm{cris}} \simeq 
%   \Ga(\mathrm{Cris}((C^\circ / (p)) / \SPEC \bZ_p) , \cO)
%   \]
%   and taking global sections induces an equivalence
%   between the category of $F$-isocrystals on
%   $\mathrm{Cris}((C^\circ / (p)) / \SPEC \bZ_p)$
%   and the category of $\varphi$-modules over $A_{\mathrm{cris}}$,
%   i.e. finite projective modules equipped with linear isomorphism
%   with their $\varphi$-twist.
%   \cite[Cor. 11.1.14]{FF18} 
% \end{prop}
% Define $B^+_{\mathrm{cris}} := A_{\mathrm{cris}}[1 / p]$.
% The relation to $X$ is as follows : 
% \begin{prop}
  
%   We have $B^+ \subs B^+_{\mathrm{cris}}$, inducing equalities 
%   \[
%     (B^+_{\mathrm{cris}})^{\varphi = p^n} 
%     = (B^+)^{\varphi = p^n} = B^{\varphi = p^n}
%   \]
%   and hence \[
%     X_{F}^{\mathrm{alg}} = \mathrm{Proj}\,B^{\varphi = p^n}
%     = \mathrm{Proj}\,(B^+)^{\varphi = p^n} 
%     = \mathrm{Proj}\,(B^+_{\mathrm{cris}})^{\varphi = p^n} 
%   \]
% \end{prop}
% \begin{proof}
%   \cite[Section 13.5]{SW20} \cite[Prop. 4.1.3]{FF18}
% \end{proof}
% Now, given $G_0$ a $p$-divisible group over $C^\circ / (p)$,
% the covariant Dieudonné module $M(G_0)$ 
% is a finite projective $A_{\mathrm{cris}}$-module.
% One can then make a vector bundle on $X_F^{\mathrm{alg}}$ by \[
%   \cE(G_0) := \text{ associated to } 
%   \bigoplus_{d \geq 0} (M[1 / p])^{\varphi = p^{d+1}}
% \]

\section{Reduction of the classiication to certain modifications}

Interestingly,
\cite[Section 5.6]{FF18}
gives a classification of vector bundles on any
``generalised Riemann sphere'',
which axiomitizes properties of $\bP^1$ needed to classify its vector bundles.

\textbf{TODO}
\begin{enumerate}
  \item classification follows from \cite[Theorem 5.6.26]{FF18}
  \item \cite[Theorem 5.6.26]{FF18} follows from \cite[Theorem 5.6.29]{FF18}
\end{enumerate}

\section{Relation to $p$-adic Hodge theory}

\textbf{TODO} \begin{enumerate}
  \item Relation of universal covers of $p$-divisible groups 
  associated to Lubin--Tate formal group laws to 
  the SES of vector bundles on $X_F$
  \[
    0 \to \cO \to \cO(1) \to i_{\infty , *} \kappa(\infty) \to 0
  \]
  Lubin--Tate formal group law in height 1 dimension 1 
  gives $r = 1 , d = 1$ hence $\cO(1)$.
  \cite[Section II.2.1]{FS24}
  \item conditions of \cite[Theorem 5.6.29]{FF18} : 
  \begin{enumerate}
    \item equivalent to surjectivity of Gross--Hopkins period morphism
    \item implied by Drinfeld upper half plane = 
    image of period morphism from Drinfeld tower
  \end{enumerate}
\end{enumerate}

\section{Other facts about the curve}

Other facts about the curve 
in the case of $S$ being a geometric point
with references from \cite{FF18}.

% \begin{dfn}
%   A Riemann sphere consists of the following : 
%   \begin{enumerate}
%     \item a complete curve $X$
%     \item a closed point $\infty \in X$ with $\deg \infty = 1$
%     \item $X \setminus \set{\infty}$ is affine
%     \item $\PIC(X \setminus \set{x}) \simeq 0$
%     \item $H^1(X , \cO_X(- \infty)) = 0$
%     where $\cO_X(-\infty)$ is the ideal sheaf of $\infty$.
%   \end{enumerate}
% \end{dfn}

\begin{prop}[Gauss norms]
  
  For $0 < \rho < 1$ a real number, define \[
    \abs{\_}_\rho : B^b_{F} \to \bR_{\geq 0} , 
    x = \sum_{n >> -\infty} [x_n] p^n 
    \mapsto \sup_{n \in \bZ} \abs{x_n}_F \rho^n
  \]
  Then we have : 
  \begin{enumerate}
    \item \cite[Prop. 1.4.3]{FF18} 
      $\abs{x + y}_\rho \leq \max(\abs{x}_\rho , \abs{y}_\rho)$
    \item \cite[Remark 1.4.4]{FF18} 
      For $\la \in \bQ_p$, $\abs{x}_\rho = \rho^{- v_p(x)}$.
    \item \cite[Prop. 1.4.9]{FF18} 
      $\abs{x y}_\rho = \abs{x}_\rho \abs{y}_\rho$.  
    \item \cite[Prop. 1.4.11.(3)]{FF18}
      The topology on $W(F^\circ)$ induced by $\abs{\_}_\rho$
      is the $(p , [\pi])$-adic topology.
      Consequently, $\abs{\_}_\rho$ is a norm on 
      $B^b_F$ as a $\bQ_p$-vector space.
    \item For $0 < a \leq b \leq c < 1$ we have 
      $\abs{x}_b \leq \max(\abs{x}_a , \abs{x}_c)$.
    \item \cite[Example 1.6.3]{FF18} 
      For $I = [\rho_1 , \rho_2] \subs (0,1)$ compact interval,
      define $B_{F , I}$ as the completion of $B^b_F$ w.r.t.
      all the norms $\abs{\_}_\rho$ with $\rho \in I$.
      By the previous point, this is equivalent to the completion
      w.r.t. just the norm 
      $\abs{\_}_I := \max(\abs{\_}_{\rho_1} , \abs{\_}_{\rho_2})$.
      If $\rho_1 = \abs{a} , \rho_2 = \abs{b}$ for $a , b \in F^\circ$
      then this completion matches what we previously computed
      \begin{align*}
        \set{x \in B^b_F \st \abs{x}_I \leq 1} = 
          W(F^\circ)\sqbrkt{\frac{[a]}{p} , \frac{p}{[b]}} \\
        B_{F , I} \simeq 
          W(F^\circ)\sqbrkt{\frac{[a]}{p} , \frac{p}{[b]}}^\wedge_p[1 / p]
      \end{align*}
      \item \cite[Lem. 2.4.7, 2.4.8]{FF18}
      Let $\abs{Y_{F}}$ 
      denote the set of characteristic zero untilts of $F$ up to isomorphism.
      For a compact interval $I \subs (0,1)$,
      let \[
        \abs{Y_{F , I}} := \set{y \in \abs{Y_F}\st r(y) \in I}
      \]
      For each $y \in \abs{Y_{F , I}}$,
      writing $\theta_y : W(F^\circ) \to K_y$ for the surjection
      to the untilt, it extends uniquely
      to a continuous surjection $\theta_y : B_{F , I} \to K_y$.
      \begin{cd}
        {W(F^\circ)} & {B^b_F} & {B_{F , I}} \\
        {K_y}
        \arrow[from=1-1, to=1-2]
        \arrow["{\theta_y}"', two heads, from=1-1, to=2-1]
        \arrow[from=1-2, to=1-3]
        \arrow[two heads, from=1-2, to=2-1]
        \arrow["{\widetilde{\theta_y}}", from=1-3, to=2-1]
      \end{cd}
      In fact, $(\ker \theta_y) B_{F , I} = \ker \widetilde{\theta_y}$.
      \item \cite[Theorem 2.5.1]{FF18} For compact intervals $I \subs (0,1)$,
      $B_{F , I}$ is a PID and 
      the previous point induces a bijection \[
        \abs{Y_{F , I}} \map{\sim}{} \mathrm{MaxSpec}\,B_{F , I}
      \]
  \end{enumerate}
\end{prop}

Recall we defined $X$ by gluing $\SPA(B_{[a,b]} , B_{[a,b]^+})$.
In fact, one can glue $\SPEC B_{[a,b]}$ in the same way
to obtain $X^{\mathrm{alg}}$.
It follows that the set of closed points can be obtained as \[
  \abs{X} \simeq \abs{Y} / \varphi^\bZ
\]
For $y \in \abs{Y}$,
$r(\varphi(y)) = r(y)^{1 / p}$.
For intervals $[a , b] \subs (0,1)$ such that 
$[a , b] \cap [a^{1 / p} , b^{1/p}] = \nothing$,
one can show that the morphism \[
  \SPA(B_{[a,b]} , B_{[a,b]^+}) \to X
\]
is an open immersion.
\textbf{Warning} : the corresponding morphism of schemes \[
  \SPEC B_{[a,b]} \to X^{\mathrm{alg}}
\]
is \emph{not} an open immersion,
because it is not the complement of finitely many points!
Nonetheless, we take the following definition : 
\begin{prop}
  
  Let $x \in \abs{X}$ and let $y \in \abs{Y}$ be a lift of $x$.
  This corresponds to an untilt $(C , \io)$ of $F$ over characteristic zero.
  Let $(\xi) = \ker \theta \subs W(F^\circ) \to C^\circ$.
  Then for any compact interval $I \subs (0,1)$
  containing $r(y)$,
  the morphism of schemes \[
    \SPEC B^b \to \SPEC B_{[a,b]} \to X^{\mathrm{alg}}
  \] 
  induces isomorphisms of rings
  \[
    \brkt{B^b}^\wedge_\xi \map{\sim}{}
    \brkt{B_{[a, b]}}^{\wedge}_{\f{m}_y} \map{\sim}{}
    \cO_{X , x}^\wedge
  \]
  We identify these rings and call it $B_{\DR}^+(C)$.
  \cite[Def. 2.7.1]{FF18}
  This is a complete DVR.
\end{prop}
\begin{proof}
  The first isomorphism is quite formal.
  The second requires more work \cite[Theorem 6.5.2.(5)]{FF18}.
  Complete DVR part follows from previous proposition.
\end{proof}

% Miscellaneous thoughts : 
% \begin{enumerate}
%   \item \cite[Def. 4.2.2.]{KL15} extended Robba ring is
%   ``punctured dagger neighbourhood of $\abs{p} = 0$''.
%   \item \cite[Prop. 2.3.10]{FF18} Lubin--Tate formal group laws
% \end{enumerate}


% We \emph{could} now use the same diagram that defines $Y_F$,
% except now in Zariski sheaves on affines over $\bQ_p$,
% and form the quotient by $\varphi$.
% This does not immediately imply $X_F^{\mathrm{alg}}$ is a scheme
% because for compact intervals $J \subsetneq I \subs (0,1)$,
% $B_{F , I} \to B_{F , J}$ is not a Zariski localization.
% An alternative idea is the following : 
% Classically, there is an equivalence or categories
% between smooth projective curves over a field $k$ and
% field extensions of $k$ with transcendence degree one,
% under which the Galois theory of fields coincide with
% the Galois theory of finite étale coverings.
% \cite{FF18} gives a construction of the schematic $X_F$
% inspired by this.

% \textbf{TODO : How can one avoid proj?}

% \begin{prop}[The schematic Fargues--Fontaine curve]
  
%   Define the field of meromorphic functions on $Y_F , X_F$ to be
%   \[
%     \cM(Y_F) := \FRAC \, \cO(Y_F) \,\,\,\,
%     \cM(X_F) := \FRAC \, \cO(X_F)
%   \]
%   Then the following are true : 
%   \begin{enumerate}
%     \item \cite[Prop. 3.5.10]{FF18} 
%     We have $\cM(Y_F) \simeq \LIM_{I} \FRAC\,B_{F , I}$
%     where $I$ ranges over compact intervals in $(0,1)$.
%     \item 
%   \end{enumerate}
% \end{prop}

\printbibliography

\end{document}